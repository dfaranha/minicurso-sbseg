\documentclass{SBCbookchapter}
\usepackage{verbatim}
\setcounter{secnumdepth}{4}
\usepackage{graphicx,url}

\usepackage{amsfonts}
\usepackage{amsmath}
\usepackage{multicol}

\usepackage[brazil]{babel}   
%\usepackage[latin1]{inputenc}  
\usepackage[utf8]{inputenc}  
% UTF-8 encoding is recommended by ShareLaTex

%==========================================================
% ToDo notes
%==========================================================
\usepackage{todonotes}
\usepackage{xargs}                      % Use more than one optional parameter in a new commands
%\usepackage[disable]{todonotes}   % uncomment for final version
% \presetkeys{todonotes}{inline}{}  %% Make todo notes inline by default.
\newcommandx{\todobase}[3][2=noinline]{\todo[fancyline,linecolor=#1,backgroundcolor=#1!25,bordercolor=#1,#2]{#3}}
%\newcommand{\myreminder}[1]{\todo[inline,linecolor=cyan,backgroundcolor=cyan!25,bordercolor=cyan]{#1}}
%\newcommand{\NSLDS}[1]{\todo[inline,linecolor=cyan,backgroundcolor=cyan!25,bordercolor=cyan]{#1}}
\newcommandx{\thiswillnotshow}[2][1=]{\todo[fancyline,disable,#1]{#2}}
\newcommandx{\erick}[2][1=]{\todobase{yellow}[#1]{#2}}
\newcommandx{\lucas}[2][1=]{\todobase{green}[#1]{#2}}
\newcommandx{\ricardo}[2][1=]{\todobase{blue}[#1]{#2}}

%==========================================================
% Finite Field
%==========================================================
\newcommand{\Fp}{$\mathbb{F}_p$}
\newcommand{\Fpfull}{$\mathbb{F}_{2^{255}-19}$}
%\ceil and \floor
\usepackage{mathtools}
\DeclarePairedDelimiter\ceil{\lceil}{\rceil}
\DeclarePairedDelimiter\floor{\lfloor}{\rfloor}
% alias for assignment arrow
\newcommand{\rcv}{\leftarrow}

%==========================================================
% Algorithms
%==========================================================
\usepackage{algorithm}
\usepackage{algorithmic}
\renewcommand{\algorithmicrequire} {\textbf{\textsc{Inputs:}}}
\renewcommand{\algorithmicensure}  {\textbf{\textsc{Outputs:}}}
\usepackage{enumitem}

%===============================================================
% Package to simplify referencing of section, figure, algorithm, etc.
% Must be loaded AFTER hyperref.
%===============================================================
\usepackage{cleveref}

%===========================================================
% Global constants for table formatting settings
\newcommand{\tblvertpaddingfactor}{1.3}
\newcommand{\tblhorizpadding}{3pt}
%===========================================================

\sloppy

%\title{Canais laterais em criptografia simétrica e de curvas elípticas: ataques e contramedidas\footnote{Esse trabalho é uma versão aprofundada e modernizada do minicurso ministrado no SBSeg 2009 por João Paulo Fernandes Ventura e Ricardo Dahab}}

\title{Canais laterais em criptografia simétrica e de curvas elípticas: ataques e contramedidas}

\author{Lucas Zanco Ladeira, Erick Nascimento, João Paulo Fernantes Ventura,\\ Ricardo Dahab, Diego F. Aranha, Julio Lopez}



\begin{document} 

\maketitle
     
\begin{resumo}
Para fornecer segurança em um ambiente hostil, algoritmos criptográficos precisam resistir a uma infinidade de ataques buscando a obtenção de informação sigilosa, acesso não-autorizado, entre outros. Tais ataques ocorrem tanto sobre os algoritmos e seus problemas computaconais subjacentes, quanto as implementações de um criptossistema. Uma classe de ataques sobre implementações de algoritmos é chamada ataques de canal lateral, que faz uso de informações vazadas durante a execução de uma primitiva criptográfica. Ataques dessa natureza utilizam variações no tempo de execução, no consumo de energia, emanações eletromagnéticas e outras características do dispositivo alvo. Contramedidas contra esses ataques podem ser baseadas em modificações no software ou no hardware. Nesse minicurso, discutem-se ataques e contramedida sobre implementações em software de métodos criptográficos simétricos e assimétricos baseados em curvas elípticas.
\end{resumo}

\begin{abstract}
In order to provide security in a hostile environment, cryptographic algorithms must resist against many attacks aiming to capture confidential information, obtain non-authorized access, among others. These attacks can target either the algorithms and their underlying hardness assumptions, and the implementations of a cryptosystem. Side-channel attacks are a class of attacks directed at the implementations of a cryptographic algorithm and make use of information leaked during the execution of a cryptographic primitive. Such attacks are based on variances in execution time, energy consumption, electromagnetic emanations and other features of the device. Countermeasures against this approach can be both based on software and hardware mechanisms. In this short course, we discuss side-channel attacks and countermeasures against software implementations of symmetric primitives and curve-based public key cryptography.
\end{abstract}


\section{Introdução}

%% Seções da introdução (rascunho)

%%%%%% Segurança da informação; criptografia; segurança em IOT

% Criptografia Simetrica / Hash / Assimetrica
% Criptografia de curvas elípticas
% Permite SCA / diferentes ataques abordados
% Ataques e contramedidas Simetrico / ECC
% Segurança em IoT

\section{Encriptação simétrica e hash}

Teste



\section{Criptografia de curvas elípticas (ECC)}
Criptografia de curvas elípticas é uma classe de algoritmos criptográficos que se baseia na aritmética de pontos de uma curva elíptica em um corpo finito $\mathbb{F}$ \cite{Hankerson:2003:GEC:940321}. Os algoritmos criptográficos que utilizam esse tipo de problema matemático podem se diferenciar de acordo com vários fatores, como por exemplo: o primo utilizado, a curva, corpo finito $\mathbb{F}_p$ ou $\mathbb{F}_{2^m}$, o mapeamento dos pontos na curva, entre outros. 

As operações mais simples executadas em uma curva é a adição de pontos e a duplicação de um ponto, sendo $R = P + Q$ e $R = P + P$ respectivamente. Dependendo da curva utilizada é possível executar essas operações de maneira mais eficiente podendo diferenciar o mapeamento desses pontos na curva.

Seguindo a notação descrita por \cite{Hankerson:2003:GEC:940321} podemos definir uma curva elíptica $E$ como: $E: y^2 + a_1xy + a_3y = x^3 + a_2x^2 + a_4x + a_6$ sobre um corpo K. Onde $\bigtriangleup \neq 0$, essa condição garante que não existirá ponto onde a curva possui duas ou mais linhas de tangente diferentes. A linha da tangente é utilizada na operação de duplicação de um ponto. Sendo que, $\bigtriangleup$ é definido como:

\begin{align*}
\bigtriangleup &= -d_2^2d_8 - 8d_4^3 - 27d_6^2 + 9d_2d_4d_6 \\
d_2 &= a_1^2 + 4a_2 \\
d_4 &= 2a_4 + a_1a_3 \\ 
d_6 &= a_3^2 + 4a_6 \\
d_8 &= a_1^2a_6 + 4a_2a_6 - a_1a_3a_4 + a_2a_3^2 - a_4^2
\end{align*}

A curva descrita anteriormente é chamada de equação de Weiertrass, onde a mesma possui uma forma simplificada: $y^2 = x^3 + ax + b$. Essa equação é utilizada como ponto inicial para descrição de várias curvas, sendo que pode-se variar os valores de $a$ e $b$ para obter diferentes curvas. Durante a escolha de qual utilizar é possível verificar a facilidade de implementação, possibilidade de paralelização, e também o desempenho da mesma. 

O desempenho está ligado as operações executadas, como por exemplo o método Montgomery-Ladder utilizado durante a multiplicação escalar, e pelas instruções do processador que facilitam a aritmética. Na seção a seguir iremos exibir as curvas com mais ocorrência na literatura, e algumas propriedades de cada uma.

\subsection{Curvas NIST e curvas modernas}
Primeiramente iremos apresentar as curvas NIST, essas curvas são recomendações de utilização feitas pelo NIST (National Institute of Standards and Technlogy), situado nos Estados Unidos da América. As mesmas foram geradas de forma pseudo-aleatória pela NSA, e possuem ao todo são 10 corpos finitos sendo 5 corpos primos ($\mathbb{F}_p$) e 5 corpos binários ($\mathbb{F}_{2^m}$) \cite{Brown2001}.

Os corpos foram recomendados com o foco no desempenho das curvas, facilitando a aritmética utilizada. Todavia existe uma resistência da comunidade em adotar o que foi proposto, pela incerteza na existência de vulnerabilidades, inseridas para obter informações secretas pelo governo norte americano. Os corpos finitos recomendados podem ser observados a seguir, sendo P os corpos primos e B as corpos binários:

\begin{multicols}{2}
\begin{itemize}
\item P-192 \\ $\mathbb{F}_{192}$ $p = 2^{192} - 2^{64} - 1$; 
\item P-224 \\ $\mathbb{F}_{224}$ $p = 2^{224} - 2^{96} + 1$;
\item P-256 \\ $\mathbb{F}_{256}$ $p = 2^{256} - 2^{224} + 2^{192} + 2^{96} - 1$;
\item P-384 \\ $\mathbb{F}_{384}$ $p = 2^{384} - 2^{128} - 2^{96} + 2^{32} - 1$;
\item P-521 \\ $\mathbb{F}_{521}$ $p = 2^{521} - 1$;
\item B-163 \\ $\mathbb{F}_{2^{163}}$ $f(x) = x^{163} + x^7 + x^6 + x^3 + 1$;
\item B-233 \\ $\mathbb{F}_{2^{233}}$ $f(x) = x^{233} + x^{74} + 1$;
\item B-283 \\ $\mathbb{F}_{2^{283}}$ $f(x) = x^{283} + x^{12} + x^7 + x^5 + 1$;
\item B-409 \\ $\mathbb{F}_{2^{409}}$ $f(x) = x^{409} + x ^{87} + 1$;
\item B-571 \\ $\mathbb{F}_{2^{571}}$ $f(x) = x^{571} + x^{10} + x^5 + x^2 + 1$.
\end{itemize}
\end{multicols}

Nesses corpos finitos primos é recomendado utilizar as curvas pseudo-aleatórias, já no caso dos corpos binários recomenda-se, além da utilização das curvas pseudo-aleatórias, o uso da curva de Koblitz. A geração de curvas pseudo-aleatórias segue três passos: primeiramente é gerada uma semente, após a partir da semente é gerada uma curva. Por fim, é verificado se a curva gerada é resistente aos ataques conhecidos, caso não seja o processo é repetido.

Existem duas curvas elípticas que estão se destacando considerando o desempenho obtido junto à técnicas de multiplicação escalar eficiente, redução modular, entre outras. Elas são a curva de Montgomery e a curva de Edwards, onde é possível encontrar implementações resistentes a ataques de canal lateral como ataque por tempo e ataque por cache.

A curva de Montgomery tem a seguinte equação: $E: y^2 = x^3 + Ax^2 + x$. O valor do parâmetro $A$ pode ser alterado para melhorar o desempenho das multiplicações escalares. Iremos tratar a utilização dessa curva com o primo $25519$ $(2^{255}-19)$ e o valor de $A = 486662$, dessa maneira a curva é definida sobre o corpo $\mathbb{F}_{2^{255}-19}$ \cite{Dull:2015:HCM:2834659.2834708}. Considerando o corpo finito utilizado essa curva é chamada de 25519, onde os pontos na curva são mapeados como $P = (X : Z)$. 

A curva de Edwards tem a fórmula: $E: -x^2 + y^2 = 1 + dx^2y^2$ \cite{Bernstein2012}. Nessa curva a soma de dois pontos segue a \textit{Edwards addition law}:
$$ (x_1,y_1) + (x_2,y_2) = (\frac{x_1y_2 + x_2y_1}{1 + xd_1x_2y_1y_2},\frac{y_1y_2 + x_1x_2}{1 - dx_1x_2y_1y_2}) $$%//TODO

Uma contramedida encontrada de fácil implementação para se tornar resistente a alguns ataques de canal lateral, como ataque de cache e hyperthreading, é possível apenas não escrever índices de dados secretos na memória RAM. Considerando que os ataques mencionados dependem das informações obtidas pela utilização dessas memórias eles não seriam efetivos. 

Um outro ataque conhecido é o ataque por tempo, o mesmo analisa o tempo de execução da primitiva criptográfico que utiliza a chave secreta, para tentar obter os bits da chave. A resistência do algoritmo está baseada na execução em tempo constante, o que pode ser obtido sem a utilização de branchs condicionais. Na seção 6 do minicurso iremos apresentar melhor essa contramedida e a eficácia da mesma.

\subsection{Algoritmos para multiplicação escalar (ECSM) básicos}

% Left-to-right binário (a.k.a. Point and Add Not-Always).
% Left-to-right binário com contramedida “Dummy Adds” de Coron  (a.k.a. Point and Add Always).
% Atomic Point and Add. Apresentar a idéia geral, e talvez um conjunto de fórmulas para uma dada curva como exemplo.
% Montgomery Ladder. Apenas a versão para curvas na forma de Montgomery.

Existem vários métodos para executar a multiplicação escalar, sendo que os mais simples de implementar serão apresentados nessa seção. O primeiro é o método \textit{Double-and-add}, o mesmo é utilizado para calcular $dP$. Ele possui quatro entradas, sendo $N$ uma variável para armazenar o resultado da operação de duplicação do ponto, Q armazena o resultado das operações, P é o ponto escolhido na curva para a multiplicação escalar, e por fim $d$ é a chave utilizada na multiplicação. É importante citar que $m$ é o tamanho da chave em bits, sendo que cada bit da chave será utilizado nesse método, totalizando 256 execuções para uma chave de 32 bytes.

\floatname{algorithm}{Algoritmo}
\begin{algorithm}[H]
\caption{Double-and-add}
\begin{algorithmic} 
    \REQUIRE $N, Q, P, d$
    \ENSURE $Q$
    \STATE $N \leftarrow P$
    \STATE $Q \leftarrow 0$
    \FOR{$i$ from $0$ \TO $m$} 
        \IF{$d_i = 1$}
            \STATE $Q \leftarrow addpoint(Q, N)$
        \ENDIF
        \STATE $N \leftarrow doublepoint(N)$
    \ENDFOR
    \RETURN Q
    \end{algorithmic}
\end{algorithm}


\subsection{Uso de tabelas precomputadas para melhorar desempenho; algoritmos de janela fixa}

\subsection{Algoritmos regulares; atomicidade; montgomery ladder}

Utilizando a curva de Montgomery é possível executar a multiplicação escalar pelo método chamado Montgomery Ladder. Esse método é dividido em degraus, sendo necessário executar 255 vezes para concluir a computação \cite{Dull:2015:HCM:2834659.2834708}.

\begin{algorithm}[H]
\caption{Montgomery ladder}
\begin{algorithmic} 
    \REQUIRE $s, x_p$
    \ENSURE $X_1, Z_1$
    \STATE $X_1 \leftarrow 1$
    \STATE $Z_1 \leftarrow 0$
    \STATE $X_2 \leftarrow x_p$
    \STATE $Z_2 \leftarrow 1$
    \STATE $p \leftarrow 0$
    \FOR{$i \leftarrow 254$ \TO $0$}
        \STATE $b \leftarrow s_i$
        \STATE $c \leftarrow b \oplus p$
        \STATE $p \leftarrow b$
        \STATE $(X_1, X_2) \leftarrow cswap(X_1,X_2,c)$
        \STATE $(Z_1,Z_2) \leftarrow cswap(Z_1,Z_2,c)$
        \STATE $(X_1, Z_1, X_2, Z_2) \leftarrow ladder(x_p, X_1, Z_1, X_2, Z_2)$
    \ENDFOR
    \RETURN $(X_1,Z_1)$
    \end{algorithmic}
\end{algorithm}

\begin{algorithm}[H]
\caption{Ladder}
\begin{multicols}{2}
\begin{algorithmic} 
    \REQUIRE $x_D, X_1, Z_1, X_2, Z_2$
    \ENSURE $X_1, Z_1, X_2, Z_2$
    \STATE $T_1 \leftarrow X_2 + Z_2$
    \STATE $X_2 \leftarrow X_2 - Z_2$
    \STATE $Z_2 \leftarrow X_1 + Z_1$
    \STATE $X_1 \leftarrow X_1 - Z_1$
    \STATE $T_1 \leftarrow T_1 \cdot X_1$
    \STATE $X_2 \leftarrow X_2 \cdot Z_2$
    \STATE $Z_2 \leftarrow Z_2 \cdot Z_2$
    \STATE $X_1 \leftarrow X_1 \cdot X_1$
    \STATE $T_2 \leftarrow Z_2 - X_1$
    \STATE $Z_1 \leftarrow T_2 \cdot a_24$
    
    
    \STATE $Z_1 \leftarrow Z_1 + X_1$
    \STATE $Z_1 \leftarrow T_2 \cdot Z_1$
    \STATE $X_1 \leftarrow Z_2 \cdot X_1$
    \STATE $Z_2 \leftarrow Z_2 - X_2$
    \STATE $Z_2 \leftarrow Z_2 \cdot Z_2$
    \STATE $Z_2 \leftarrow Z_2 \cdot x_D$
    \STATE $X_2 \leftarrow T_1 + X_2$
    \STATE $X_2 \leftarrow X_2 \cdot X_2$
    
    \RETURN $(X_1,Z_1)$
\end{algorithmic}
\end{multicols}
\end{algorithm}

\subsection{Protocolos de IoT que envolvem ECC}

\section{Ataques por canais laterais}
\subsection{Canal lateral (SCA) de tempo}

\subsubsection{O que é? Por que é importante?}
A premissa fundamental de ataques temporais \'{e} que o tempo gasto na execu\c{c}\~{a}o de uma instru\c{c}\~{a}o \'{e} influenciado por seus respectivos operandos \cite{ECCBook_HankersonVanstone2004}. Estudos mostraram \cite{1251354} a viabilidade desse ataque contra servidores executando protocolos como o SSL com RSA devido à lat\^{e}ncia da comunica\c{c}\~{a}o decorrente da rede local.

Como todos os ataques por canais laterais envolvem o monitoramento de uma grandeza f\'{i}sica, a limita\c{c}\~{a}o do \textit{time attack} \'{e} que as opera\c{c}\~{o}es com a chave criptogr\'{a}fica sejam \textit{lentas} o suficiente para serem medidas.

Acreditava-se que esse tipo de ataque seria poss\'{i}vel apenas em opera\c{c}\~{o}es de rede ou r\'{a}dio e jamais em processadores de disposit\'{i}vos m\'{o}veis ou se quer computadores justamente devido essa limita\c{c}\~{a}o.

Por\'{e}m uma forma derivada desse ataque denominada \textit{branch prediction analysis} (an\'{a}lise de preditor de salto) \cite{1266999} demonstrou ser poss\'{i}vel atacar uma implementa\c{c}\~{a}o do OpenSSL rodando em processadores convencionais (PowerPC, Intel, ARM, etc.).

Executando processos maliciosos em sistemas operacionais (Windows, Linux, Android, iOS, BlackBerry, etc.), foi demonstrado ser poss\'{i}vel afetar a execu\c{c}\~{a}o do OpenSSL. Tornando as itera\c{c}\~{o}es da exponencia\c{c}\~{a}o modular mais lentas, passando de nanosegundos para microsegundos, podem ser detecdadas e informações sobre a chave privada inferida.

\subsubsection*{Unidade de predi\c{c}\~{a}o de saltos}

As instru\c{c}\~{o}es que comp\~{o}em o c\'{o}digo bin\'{a}rio de um programa execut\'{a}vel podem consumir diferentes quantidades de ciclos de \textit{clock} de acordo com suas respectivas complexidades. Como no decorrer do fluxo de programas podem existir diversas depend\^{e}ncias entre as instru\c{c}\~{o}es executadas, existe a possibilidade de que valores necess\'{a}rios para a execu\c{c}\~{a}o de uma determinada instru\c{c}\~{a}o ainda n\~{a}o tenham sido calculados.

Quando a instru\c{c}\~{a}o depende um salto condicional, ent\~{a}o essa situa\c{c}\~{a}o \'{e} denominada \textit{control hazard}. Para que o processador n\~{a}o permane\c{c}a ocioso at\'{e} que o fluxo do programa seja definido, durante o per\'{i}odo de decis\~{a}o ele especula qual dever\'{a} ser a pr\'{o}xima instru\c{c}\~{a}o executada. Se a predi\c{c}\~{a}o se mostrar correta (\textit{hit}) o fluxo do programa prossegue sem degrada\c{c}\~{a}o de desempenho; caso a predi\c{c}\~{a}o se mostre incorreta (\textit{miss prediction}), o \textit{pipeline} deve ser esvaziado e a instru\c{c}\~{a}o correta tomada. Observe que uma \textit{miss prediction} acarreta em uma penalidade de ciclos de \textit{clock} que \'{e} proporcional \`{a} quantidade de est\'{a}gios do \textit{pipeline}.

Quando a CPU determina um salto como tomado, ela deve buscar a instru\c{c}\~{a}o do endere\c{c}o alvo do salto na mem\'{o}ria e entreg\'{a}-la a unidade de execu\c{c}\~{a}o. Para tornar o processo mais eficiente, a CPU mant\'{e}m um registro dos saltos executados anteriormente no BTB (\textit{Branch Target Buffer}). Observe que o tamanho do BTB \'{e} limitado; logo, alguns endere\c{c}os armazenados precisam ser expulsos para que novos endere\c{c}os sejam armazenados.
O preditor tamb\'{e}m possui uma parte denominada BHR (\textit{Branch History Registers}) respons\'{a}vel por gravar a hist\'{o}ria dos registradores usados globalmente e localmente pelo programa. \cite{Jean-Pierre06predictingsecret}.

\begin{figure}[ht]
	\centering
	\includegraphics[width=1\textwidth]{figures/automato.png}
	\caption{Aut\^{o}mato finito descreve o comportamento do preditor de saltos \cite{493986}.}
	\label{fig:Fig_automato}
\end{figure}

\begin{figure}[ht]
	\centering
	\includegraphics[width=.5\textwidth]{figures/btu.jpg}
	\caption{Unidade de predi\c{c}\~{a}o de saltos \cite{Jean-Pierre06predictingsecret}.}
	\label{fig:Fig_btu}
\end{figure}

\subsubsection*{Medi\c{c}\~{a}o direta de tempo}

A m\'{a}quina de estados que descreve as poss\'{i}veis decis\~{o}es da BTU possui um n\'{u}mero finito de estados; logo, o algoritmo que a descreve \'{e} determin\'{i}stico. O advers\'{a}rio pode assumir que a implementa\c{c}\~{a}o do RSA utilizou S\&M (\textit{Square-and-Multiply exponentiation algorithm}) e MM (\textit{Montgomery Multiplication algorithm} \cite{ECCBook_HankersonVanstone2004, 1197338}) e o BTU possui um aut\^{o}mato finito de apenas dois estados: salto tomado ou n\~{a}o tomado.

Seja $d$ a chave privada, vamos supor que o advers\'{a}rio conhece seus $i$ primeiros bits e est\'{a} tentando determinar $d_{i}$. Para qualquer mensagem $m$, o advers\'{a}rio pode simular as primeiras $i$ itera\c{c}\~{o}es e obter um resultado intermedi\'{a}rio que ser\'{a} a entrada da $(i+1)-$\textit{\'{e}sima} itera\c{c}\~{a}o. Ent\~{a}o ele gera quatro conjuntos distintos tais que:

\begin{align} \notag
	M_{1} = \left\lbrace m\ \vert\ d_{i} = 1 \rightarrow m\ causa\ missprediction\ durante\ MM\right\rbrace \\ \notag
	M_{2} = \left\lbrace m\ \vert\ d_{i} = 1 \rightarrow m\ causa\ hit\ durante\ MM\ \ \ \ \ \ \ \ \ \ \ \ \ \ \ \ \ \ \ \ \right\rbrace \\ \notag
	M_{3} = \left\lbrace m\ \vert\ d_{i} = 0 \rightarrow m\ causa\ missprediction\ durante\ MM\right\rbrace \\ \notag
	M_{4} = \left\lbrace m\ \vert\ d_{i} = 0 \rightarrow m\ causa\ hit\ durante\ MM\ \ \ \ \ \ \ \ \ \ \ \ \ \ \ \ \ \ \ \ \right\rbrace \\ \notag
\end{align}

O advers\'{a}rio calcula o tempo m\'{e}dio de execu\c{c}\~{a}o na multiplica\c{c}\~{a}o de Montgomery em cada conjunto $M_{i}$. Sendo $d_{i} = t, t \in \left\lbrace 0,1\right\rbrace $, a diferen\c{c}a dos tempos m\'{e}dios de execu\c{c}\~{a}o para o mesmo valor correto $t$ ser\~{a}o muito mais significativas do que a obtida dos outros dois conjuntos, pois, para o valor incorreto, os valores de tempo de cada multiplica\c{c}\~{a}o ter\~{a}o um caracter aleat\'{o}rio. Esse \'{e} o mesmo processo estat\'{i}stico da an\'{a}lise diferencial de pot\^{e}ncia. Portanto, se a diferen\c{c}a entre os tempos m\'{e}dios de $M_{1}$ e $M_{2}$ for muito mais significativa do que $M_{3}$ e $M_{4}$, ent\~{a}o o palpite correto \'{e} $d_{i} = 1$, e $d_{i} = 0$ caso contr\'{a}rio. 

Nesse ataque o advers\'{a}rio precisa saber de antem\~{a}o o estado do BPU antes do algoritmo de decripta\c{c}\~{a}o ser iniciado. Uma possibilidade de simples implementa\c{c}\~{a}o, por\'{e}m menos eficiente, seria realizar a an\'{a}lise supondo cada um dos quatro estados iniciais. A segunda abordagem consiste em for\c{c}ar o estado inicial do BPU de modo que nenhum endere\c{c}o de salto esteja no BTB. Essa abordagem ser\'{a} fundamentalmente a mesma utilizada em todos os ataques de predi\c{c}\~{a}o de salto listados a seguir.

\subsubsection*{For\c{c}ando BPU \`{a} mesma predi\c{c}\~{a}o assincronamente}

Unidades de processamento que permitem execu\c{c}\~{a}o concorrente de processos (SMT ou \textit{Simultaneous Multi-Threading} \cite{Silberschatz2004}) permitem que um advers\'{a}rio execute um processo espi\~{a}o simultaneamente ao programa de encripta\c{c}\~{a}o. Dessa forma, o advers\'{a}rio pode fazer com que o valor previsto dos saltos do encriptador nunca estejam no BTB; conseq\"{u}entemente, sempre ocorrer\'{a} um \textit{missprediction} quando o resultado correto, segundo a previs\~{a}o, seria que o salto fosse tomado. Comparado ao processo anterior, a an\'{a}lise diferencial seria similiar exceto pelo fato de que $d_{i} = 1$ em caso de \textit{hit} e $d_{i} = 0$ em caso de \textit{missprediction} durante o c\'{a}lculo de $m^{2} \bmod N$.

O processo espi\~{a}o remover do BTB o endere\c{c}o alvo de salto dos seguintes modos:
\begin{enumerate}
	\item (\textit{Total Eviction Method}: todas as entradas do BTB s\~{a}o expulsas.
	\item (\textit{Partial Eviction Method}): um conjunto de entradas do BTB \'{e} expulso.
	\item (\textit{Single Eviction Method}): apenas endere\c{c}o de interesse \'{e} expulso da tabela.
\end{enumerate}

Obviamente o primeiro m\'{e}todo \'{e} o de mais simples implementa\c{c}\~{a}o (assumindo que sejamos capazes de esvaziar todo o BTB entre duas itera\c{c}\~{o}es da exponencia\c{c}\~{a}o). O diferencial desse ataque \'{e} o advers\'{a}rio n\~{a}o ter que saber detalhes de implementa\c{c}\~{a}o da BPU para ser capaz de criar o processo espi\~{a}o e determinar quais s\~{a}o os bits da chave secreta.

\begin{figure}[ht]
	\centering
	\includegraphics[width=.7\textwidth]{figures/totaleviction.jpg}
	\caption{Resultados pr\'{a}ticos do \textit{Total Eviction Method} \cite{Jean-Pierre06predictingsecret}.}
	\label{fig:Fig_totaleviction}
\end{figure}

Esse ataque foi aplicado sobre uma implementa\c{c}\~{a}o do RSA em OpenSSL vers\~{a}o 0.9.7, rodando sob uma workstation RedHat 3. Foram gerados 10 milh\~{o}es de blocos de mensagens aleat\'{o}rias e chaves aleat\'{o}rias de $512$ bits. As mensagens foram encriptadas e separadas segundo os crit\'{e}rios acima, sendo assumido como tomado o salto do pr\'{o}ximo bit desconhecido.
       
Na Figura ~\ref{fig:Fig_totaleviction} (a), o eixo $x$ corresponde aos bits do expoente de 2 at\'{e} 511, sendo que cada coordenada $x_{i}$ apresenta os valores das m\'{e}dias das separa\c{c}\~{o}es correta e a m\'{e}dia das separa\c{c}\~{o}es aleat\'{o}rias, denotadas respectivamente por $\mu_{Y_{i}}$ e $\mu_{X_{i}}$. Analizando todos os pares $(\mu_{Y_{i}}, \mu_{X_{i}})$, o advers\'{a}rio verifica  qual deles teve a diferen\c{c}a mais significativa (Figura ~\ref{fig:Fig_totaleviction} (b)) e utiliza seus respectivos desvios padr\~{o}es para determinar o desvio da diferen\c{c}a das m\'{e}dias

\begin{align}\notag
    \mu_{Z} &= \mu_{Y} - \mu_{X} = 58.91 - 1.24 = 57.67\\\notag
    \sigma_{Z} &= \sqrt{ \sigma_{Y}^{2} + \sigma_{X}^{2}} = \sqrt{ 62.58^{2} - (34.78)^{2}} = 71.60\\ \notag
\end{align}

Sempre que o advers'{a}rio encontrar $Z > 0$, ele ir\'{a} supor que seu palpite do valor do \textit{bit} foi correta. O grau de certeza que o advers\'{a}rio pode ter nessas decis\~{o}es pode ser medido atrav\'{e}s da probabilidade:

\begin{align}
    Pr[Z > 0] = \phi(\dfrac{0 - \mu_{Z}}{\sigma_{Z}}) = \phi(-0.805) = 0.79\notag
\end{align}

Portanto, a probabilidade de suas decis\~{o}es estarem corretas para essas medidas \'{e} de quase 80\%, viabilizando o advser\'{a}rio obter o restante da chave por for\c{c}a bruta.

\subsubsection{Limitações}

\subsubsection{Impacto em IoT}

\subsection{Canal lateral de potência}

\subsubsection{Características gerais}
Em um ataque sobre o canal lateral de pot\^{e}ncia, o advers\'{a}rio analisa sutis varia\c{c}\~{o}es no consumo de energia el\'{e}trica de um dispositivo cujo \textit{hardware} implementa um algoritmo criptogr\'{a}fico (sensores RFID, \textit{smartcards}, \textit{SIM cards}, etc).

Opera\c{c}\~{o}es com dados sens\'{i}veis geram alter\c{c}\~{o}es na corrente ou tens\~{a}o da alimenta\c{c}\~{a}o do dispositivo, permitindo extrair parcialmente (ou mesmo integralmente) a chave criptogr\'{a}fica e outras informa\c{c}\~{o}es  sens\'{i}veis. O primeiro ataque dessa natureza foi apresentado por \cite{Kocher:1999:DPA:646764.703989}, tamb\'{e}m autor da c\'{e}lebre pesquisa precursora sobre \textit{time attacks} \cite{Kocher96}. 

\paragraph{Comparação entre canais laterais de potência e de tempo}

\subsubsection{Ataque de potência simples (SPA)}
A tecnologia de semicondutores dominante em microprocessadores, mem\'{o}rias e dispositivos embarcados \'{e} a CMOS   \cite{sedra:1997}, sendo inversores l\'{o}gicos sua unidade b\'{a}sica de constru\c{c}\~{a}o. Como dispositivos utilizam fontes constantes de tens\~{a}o, a pot\^{e}ncia consumida varia de acordo com o fluxo de sinais nos componentes, e esses de acordo com as opera\c{c}\~{o}es realizadas. Se esse consumo de pot\^{e}ncia for monitorado com aux\'{i}lio de um oscilosc\'{o}pio poderemos estabelecer um rastro de consumo de pot\^{e}ncia (\textit{power trace}) a cada ciclo do dispositivo.

\subsubsection{An\'{a}lise simples de pot\^{e}ncia sobre ECDSA}
Uma das rotinas mais executadas em dispositivos que utilizam ECC s\~{a}o os algoritmos de assinatura digital de curvas el\'{i}pticas (\textit{ECDSA} ou \textit{Elliptic Curve Digital Signature Algorithm}), tendo como opera\c{c}\~{a}o central a multiplica\c{c}\~{a}o de um ponto por um escalar (Algoritmo 8).

\floatname{algorithm}{Algoritmo}
\begin{algorithm}[H]
\caption{M\'{e}todo NAF bin\'{a}rio de multiplica\c{c}\~{a}o escalar de um ponto}
\begin{algorithmic}
    \REQUIRE $k \in \mathbb{N}$
    \REQUIRE $P \in E(\mathbb{F}_p)$
    \ENSURE $Q = kP \in E(\mathbb{F}_p)$
    \STATE $(k_{l-1}, k_{l-2}, ..., k_{1}, k_{0}) \leftarrow NAF(k)$
    \STATE $Q \leftarrow \infty$
    \FOR {$j \leftarrow l - 2$ \TO $0$}
        \STATE $Q \leftarrow 2Q$
        \IF {$k_{i} = 1$}
            \STATE $Q \leftarrow Q + P$
        \ENDIF
        \IF {$k_{i} = -1$}
            \STATE $Q \leftarrow Q - P$
        \ENDIF
    \ENDFOR
    \RETURN $Q$
    \end{algorithmic}
\end{algorithm}

\begin{figure}[ht]
	\centering
	\includegraphics[width=.8\textwidth]{figures/spa1.jpg}
	\caption{Consumo de pot\^{e}ncia durante c\'{a}lculo de $kP$ \cite{ECCBook_HankersonVanstone2004}.}
	\label{fig:Fig5}
\end{figure}

O que torna a forma n\~{a}o adjacente de $k$ mais interessante do que sua representa\c{c}\~{a}o bin\'{a}ria \'{e} o fato da $NAF(k)$ possuir apenas $1/3$ de d\'{i}gitos n\~{a}o nulos. Conseq\"{u}entemente uma quantidade muito menor de adi\c{c}\~{o}es (linhas $04$ e $05$ do Algoritmo 8) s\~{a}o efetuadas.

Entretanto um advers\'{a}rio que soubesse que o dispositivo implementa um algoritmo \textit{ECDSA} poderia monitorar o consumo de pot\^{e}ncia do dispositivo utilizando um oscilosc\'{o}pio, obtendo o gr\'{a}fico mostrado na Figura~\ref{fig:Fig5}. No Algoritmo 9, vemos que adi\c{c}\~{o}es s\~{a}o realizadas apenas quando $k_{i} \neq 0$; logo, uma maior quantidade de pot\^{e}ncia \'{e} despendida para d\'{i}gitos n\~{a}o nulos. Portanto os intervalos curtos denominados $D$ correspondem a itera\c{c}\~{o}es em que $k_{i} = 0$, enquanto intervalos longos denominados $S$ correspondem a itera\c{c}\~{o}es em que $k_{i} \neq 0$. Essa informa\c{c}\~{a}o torna vi\'{a}vel descobrir a chave atrav\'{e}s de ataques por for\c{c}a bruta, pois apenas $1/3$ dos d\'{i}gitos s\~{a}o n\~{a}o nulos.

A solu\c{c}\~{a}o mais simples contra SPA consiste em inserir opera\c{c}\~{o}es redundantes no algoritmo de multiplica\c{c}\~{a}o (Algoritmo 1.6), de modo que a seq\"{u}\^{e}ncia de opera\c{c}\~{o}es elementares envolvidas sejam realizadas em igual propor\c{c}\~{a}o. Comparando o novo \textit{power trace} obtido (Figura~\ref{fig:Fig7}) n\~{a}o \'{e} poss\'{i}vel diferenciar adi\c{c}\~{o}es de multiplica\c{c}\~{o}es.

\floatname{algorithm}{Algoritmo}
\begin{algorithm}[H]
\caption{M\'{e}todo NAF bin\'{a}rio de multiplica\c{c}\~{a}o escalar resistente \`{a} SPA}
\begin{algorithmic}
    \REQUIRE $k \in \mathbb{N}$
    \REQUIRE $P \in E(\mathbb{F}_p)$
    \ENSURE $Q = kP \in E(\mathbb{F}_p)$
    \STATE $(k_{l-1}, k_{l-2}, ..., k_{1}, k_{0}) \leftarrow NAF(k)$
    \STATE $Q \leftarrow \infty$\\
    \FOR {$i = l-1$ \TO $0$}
        \STATE $Q_{0} = 2Q_{0}$
        \STATE $Q_{1} = Q_{0} + P$
        \STATE $Q_{0} = Q_{k_{i}}$
    \ENDFOR
    \RETURN $Q$
    \end{algorithmic}
\end{algorithm}

\begin{figure}[ht]
	\centering
	\includegraphics[width=.8\textwidth]{figures/spa2.jpg}
	\caption{Consumo de pot\^{e}ncia durante c\'{a}lculo de $kP$ \cite{ECCBook_HankersonVanstone2004}.}
	\label{fig:Fig7}
\end{figure}

\subsubsection{Ataque de potência diferencial (DPA)}
Quando a varia\c{c}\~{a}o do consumo de pot\^{e}ncia n\~{a}o \'{e} sens\'{i}vel o suficiente em rela\c{c}\~{a}o as opera\c{c}\~{o}es executadas por um dispositivo, o advers\'{a}rio pode monitorar como o consumo varia em rela\c{c}\~{a}o ao valor de uma determinada vari\'{a}vel. Nesse ataque, primeiramente detectamos uma vari\'{a}vel $V$, influenciada, durante um processo de decripta\c{c}\~{a}o ou assinatura digital, por um texto $m$ e uma por\c{c}\~{a}o desconhecida  da chave privada. A partir disso, definimos a fun\c{c}\~{a}o de sele\c{c}\~{a}o $V = f(k',m)$.

O advers\'{a}rio ent\~{a}o coleta milhares de \textit{power traces}, determinando indutivamente todos os bits que comp\~{o}em a chave privada atrav\'{e}s do c\'{a}lculo da derivada dessa fun\c{c}\~{a}o. Para cada bit $k'_{i}$ corretamente previsto obtemos uma derivada n\~{a}o nula para os valores de $k'$ e $m$, caso contr\'{a}rio a derivada \'{e} nula. O processo \'{e} repetido at\'{e} que cada $k'_{i}$ seja determinando \cite{ECCBook_HankersonVanstone2004}. Esse modelo de ataque \'{e} conhecido como An\'{a}lise Diferencial de Pot\^{e}ncia (DPA ou \textit{Differential Power Analysis}).


\subsection{An\'{a}lise diferencial de pot\^{e}ncia sobre ECDSA}
Ainda que o Algoritmo 10 tenha sido adotado, podemos aplicar um DPA sobre o processo de ECDSA. 

Determinada uma vari\'{a}vel $V$ cujo valor influencie o consumo de pot\^{e}ncia e uma fun\c{c}\~{a}o de sele\c{c}\~{a}o $f$ tal que $V = f(k', m)$ o advers\'{a}rio coleta milhares de \textit{power traces}, estima o tamanho que a por\c{c}\~{a}o $k'$ ocupa na chave privada e separa os dados coletados em dois grupos de acordo com o valor previsto de $V$.

No algoritmo de multiplica\'{c}\~{a}o de pontos da curva el\'{i}ptica (Algoritmo 1.6), suponha que Eve colete \textit{power traces} durante os c\'{a}lculos $kP_{1} , kP_{2} , ..., kP_{r}$ . Como $P_{1} , P_{2} , ..., P_{r}$ s\~{a}o p\'{u}blicos, ele precisa determinar apenas $k$.

\begin{center}
    \begin{tabular}{|c|c|c|c|c|}
	    \hline
		    \   & $Q_{0}$  & $Q_{0}$ & $k_{t-1}$ & $Q_{0} \leftarrow Q_{k_{t-1}}$\\
	    \hline
	        $1$ & $\infty$ &     $P$ &       $1$ & $P$\\
	    \hline
		    $2$ & ... & ... & ... & ...\\
	    \hline
		    $3$ & ... & ... & ...& ... \\
	    \hline
		    ... & ... & ... & ...& ... \\
	    \hline
    \end{tabular}

    Tabela 1.2. $k = (1, k_{t-2}, k_{t-3}, ..., k_{1}, k_{0})$.
\end{center}

Dado $Q_{0} = \infty$, o passo 2.1 \'{e} trivial e pode ser disting\"{u}ir da de uma opera\c{c}\~{a}o n\~{a}o trivial
atrav\'{e}s do power trace, logo o advers\'{a}rio pode facilmente identificar o bit mais a esquerda cujo valor e 1. Tomando $k_{t-1}= 1$, na segunda itera\c{c}\~{a}o do algoritmo temos que $Q_{0} = 2P$ (se $k_{t-2} = 0$) ou $Q_{0} = 3P$ (se $k_{t-2} = 1$).

\begin{center}
    \begin{tabular}{|c|c|c|c|c|}
	    \hline
		    \   & $Q_{0}$  & $Q_{0}$ & $k_{t-1}$ & $Q_{0} \leftarrow Q_{k_{t-1}}$\\
	    \hline
	        $1$ & $\infty$ &     $P$ &       $1$ & $P$\\
	    \hline
		    $2$ & $2P$ & $4P$ & $\color{blue}{?}$ & $\color{blue}{?}$ \\
	    \hline
		    $3$ & ... & ... & ...& ... \\
	    \hline
		    ... & ... & ... & ...& ... \\
	    \hline
    \end{tabular}

    Tabela 1.3. $k = (1, k_{t-2}, k_{t-3}, ..., k_{1}, k_{0})$.
\end{center}

Conseq\"{u}entemente, na terceira itera\c{c}\~{a}o, o valor $4P$ ser computado apenas se $k_{t-2} = 0$. Definindo $k' = k_{t-2}$ e $m = P_{i}$ ($i$-\'{e}simo bit do ponto $4P = (4P_{1} , 4P_{2} , ..., 4P_{i} , ..., 4P_{r} , )$), a fun\c{c}\~{a}o seletora calcula o valor do bit $4P_{i}$.

\begin{center}
    \begin{tabular}{|c|c|c|c|c|}
	    \hline
		    \   & $Q_{0}$  & $Q_{0}$ & $k_{t-1}$ & $Q_{0} \leftarrow Q_{k_{t-1}}$\\
	    \hline
	        $1$ & $\infty$ &     $P$ &       $1$ & $P$\\
	    \hline
		    $2$ & $2P$ & $4P$ & $\color{red}{0}$ & $2P$ \\
	    \hline
		    $3$ & $\color{red}{4P}$ & $6P$ & ...& ... \\
	    \hline
		    ... & ... & ... & ...& ... \\
	    \hline
    \end{tabular}

    Tabela 1.4. $k = (1, \color{red}{0}$$,\ $$k_{t-3}, ..., k_{1}, k_{0})$.
\end{center}

Se o gr\'{a}fico do consumo de pot\^{e}ncia da fun\c{c}\~{a}o apresentar picos, ent\~{a}o $k_{t-2} = 0$, caso contr\'{a}rio $k_{t-2} = 1$.
Esse processo \'{e} repetido at\'{e} todos os bits de $k$ serem determinados \cite{ECCBook_HankersonVanstone2004}.

Se a curva el\'{i}ptica for gerada sobre um $\mathbb{F}_{p}$ de caracter\'{i}stica superior a 3, podemos usar um sistema misto de representa\c{c}\~{a}o de coordenadas no qual $P$ seja representado em um sistema de coordenadas afins, enquanto $Q_{0}$ e $Q_{1}$ s\~{a}o representados em coordenadas jacobianas \cite{ECCBook_HankersonVanstone2004}.

Se $P = (x,y)$ no sistema afim, ap\'{o}s a primeira atribui\c{c}\~{a}o $Q_{1} \leftarrow P$ ter\'{i}amos $ Q_{1} = (x : y : 1)$. Ent\~{a}o, $Q_{1}$ seria aleatorizado com $(\lambda^{2}x, \lambda^{3}y, \lambda)$ e o algoritmo procederia como o usual. Desse modo o advers\'{a}rio estaria impedido de realizar predi\c{c}\~{o}es baseadas no valor de um bit espec\'{i}fico $4P_{i}$ em sistemas de coordenadas jacobianas aleatorizadas.

\subsubsection{High-order DPA}

\subsubsection{Template attacks}

\section{Ataques e contramedidas para encriptação simétrica e hash}
\subsection{Utilização de CPA em implementação do AES}

\subsection{Contramedida de mascaramento}

\subsection{Ataques e contramedidas para MACs baseados em funções de
hash}

\section{Ataques e contramedidas para ECC}
\subsection{Níveis em que os ataques SCA podem ser aplicados}

\begin{comment}
\subsubsection{Operações na curva}
\subsubsection{Operações no protocolo criptográfico}
\subsubsection{Transferência da chave entre diferentes memórias}
\subsubsection{Protocolo em nível de aplicação}
\end{comment}

%TODO 1 parag. p/ explicar a figura

%TODO: Figura com piramide de ataques SCA.

%TODO: "Transferência da chave entre diferentes memórias"
%TODO: Encontrar uma ref. sobre um SCA bem sucedido deste tipo

%TODO: Protocolo em nivel de aplicacao

\subsection{Ataques de tempo e SPA ao algoritmo double-and-add-not-always}

\subsubsection{Ataque de tempo ao algoritmo double-and-add-not-always}
\erick[inline]{Lucas: ver o comentario no fonte. Converter comentario em texto}.

\begin{comment}
1.	Ataque SPA à alg. ECSM binário left-to-right (Dbl-and-Add not Always)
	a.	Se impl não é de tempo constante, então é possível realizar ataque de tempo.
		i.	P.ex., se usa if and else, então pode-se determinar a cada iteração qual bloco, if ou else, é tomado.
	b. O ataque de tempo em~\cite{Kocher96} ao RSA pode ser aplicado no contexto de ECC. Segue abaixo a ideia do ataque (baseada no survey de ~\cite{Danger2013}, Sec. 3.2.1).
	
	O atacante coleta o tempo de execução de diferentes ECSMs com o mesmo escalar e diferentes pontos base. Para cada ECSM, ele simula a computação usando um simulador de software com exatamente a mesma implementação do chip alvo, "chutando" o valor do bit $i$ do escalar. Suponha, sem perda de generalidade, que a hipótese é de que o valor do bit é 0.  Ele separa os diferentes tempos de execução em dois conjuntos, S_1 e S_2. Se, a iteração
	
	%TODO: CONT HERE
	

[Kocher96] Kocher, P.C.: Timing attacks on implementations of Diffie–
Hellman, RSA, DSS, and other systems. In: Proceedings of CRYPTO’96, LNCS, vol. 1109. Springer, Berlin, pp. 104–113
(1996)
\end{comment}

\subsubsection{Ataque SPA ao algoritmo double-and-add-not-always}
\erick[inline]{Lucas: ver o comentario no fonte. Converter comentario em texto}.

\begin{comment}
b.	Se é de tempo constante, SPA (com ou sem power model) pode ser aplicado para distinguir os padrões no trace das iterações com apenas DBL (bit=0) daquelas com DBL+ADD (bit=1). Um ataque deste tipo segmenta/divide o trace de potencia de uma execucao do ECSM em subtraces, cada um contendo uma operacao de ponto (ADD ou DBL). Se as tempo(ADD) != tempo(DBL), então o comprimento dos subtraces revela onde estão os ADDs e consequentemente os bits do escalar. Se tempo(ADD) == tempo(DBL) e uma formula unificada para ADD and DBL é utilizada, então, se for aplicada correlação (coeficiente de correlacao de Pearson) entre todos os pares de subtraces, o resultado será que a correlação será mais alta para os pares de subtraces cuja operação correspondente é a mesma (i.e., (ADD,ADD) ou (DBL,DBL)), identificando portanto as operações de ponto e consequentemente os bits do escalar.
\end{comment}

%TODO 2. Argumentar que é fortemente desejável que as implementações sejam de tempo constante, de modo a ser uma base para a implementação das outra contramedidas para power analysis. 

\subsection{Double-and-add-always algorithm of Coron~\cite{Coron1999}}

\erick[inline]{Lucas: traduzir esta secao}The {\it double-and-add-always} algorithm of Coron~\cite{Coron1999} (Algorithm~\ref{Double-and-add-Coron}) uses a dummy point addition when the scalar bit $k_i$ is $0$, such that the sequence of operations to compute the scalar multiplication is independent of the value of the secret scalar.

\begin{algorithm}[h] %\scriptsize %\footnotesize
	\caption{\small{\textit{Double-and-add always} algorithm resistant against SPA}}
	\label{Double-and-add-Coron}
	\begin{algorithmic}[1]
		\REQUIRE  Point $\textbf{P} \in E(\mathbb{F}_q),$ $k=(k_{n-1},\ldots,k_1,k_0)_2 \in \mathbb{N}$
		\ENSURE  $Q=[k] \cdot P$\\
		\STATE $R_0\leftarrow P_{\infty}$   \\
		\FOR{$i$ \textbf{from} $n-1$ \textbf{to} $0$} 
		\STATE $R_0\leftarrow 2R_0$  \\
		\STATE $R_1\leftarrow R_0+P$\\ \label{Paso_R_1_Double-and-add-Coron}
		\STATE $R_0\leftarrow R_{k_i}$\label{Step5Double-and-add-Coron} \\
		\ENDFOR
		\STATE return $R_0$\\
	\end{algorithmic}
\end{algorithm}

Therefore an adversary cannot, in principle, guess the value of bit $k_i$ by SPA. A drawback of this method is its low efficiency. It requires $nA+nD$ field operations, a $33\%$ increase in the amount of field operations in comparison to the (unprotected) binary left-to-right algorithm.

\subsubsection{Fouque and Valette's Doubling Attack \cite{CHES:FouVal03}}\label{Fouque-Valette-DoublingAttack}
The doubling attack of Fouque-Valette~\cite{CHES:FouVal03} is based on the fact that it is possible to detect if two intermediate values are equal when the algorithm computes the scalar multiplication for points chosen points $P$ and $2P.$ Several algorithms protected against SPA are vulnerable to Fouque and Valette's attack, such as the classic binary left-to-right algorithm, including those derived from it, such as Coron's double-and-add-always algorithm.

%In ~\textit{double-and-add-always} algorithm (Algorithm~\ref{Double-and-add-Coron}), the partial sums are computed as follows: $S_k(P) = \sum_{i=0}^{k}d_{n-i}2^{k-i} P=\sum_{i=0}^{k-1}d_{n-i}2^{k-1-i}(2P)+d_{n-k}P= S_{k-1}(2P)+d_{n-k}P$. So, the intermediate result of the algorithm at step $k$ when given input $P$ will be equal to the intermediate result at step $k-1$ when given input $2P$, if and only if, $d_{n-k}=0$. Therefore, an attacker can obtain the secret scalar by comparing the doubling computation at step $k+1$ for $P$ and at step $k$ for $2P$ to recover the bit $d_{n-k}.$ If both computations are identical, $d_{n-k}=0$, otherwise $d_{n-k}=1$.
In Coron's double-and-add-always algorithm (Algorithm~\ref{Double-and-add-Coron}), the partial sums are computed as follows: $S_m(P) = \sum_{i=1}^{m}k_{n-i}2^{m-i} P=\sum_{i=1}^{m-1}k_{n-i}2^{m-1-i}(2P)+k_{n-m}P= S_{m-1}(2P)+k_{n-m}P$. So, the intermediate result of the algorithm at step $m$ when given input $P$ will be equal to the intermediate result at step $m-1$ when given input $2P$, if and only if, $k_{n-m}=0$. Therefore, an attacker can obtain the secret scalar by comparing the doubling computation at step $m+1$ for $P$ and at step $m$ for $2P$ to recover the bit $k_{n-m}.$ If both computations are identical, $k_{n-m}=0$, otherwise $k_{n-m}=1$. It has been shown that with only two scalar multiplication requests chosen by the attacker, it is possible to recover all the bits of the scalar.~\footnote{The attacker collects one power trace for the computation of $kP$ and one for the computation of $k(2P)$. For each iteration $m=1,...,n$, he runs the attack as described and finds $k_{n-m}$.}
%%\vspace{-0.5cm}


\subsection{Contramedidas}

% TODO: apresentar uma contramedida de cada vez entre subseções com ataques, mostrando como ela protege contra um determinado ataque.

\begin{comment} % === CONTRAMEDIDAS ===
	a.	CM1 - Scalar Randomization (SR)
	b.	CM2 - Proj. Coord. Randomization and Re-randomization (CRR)
	c.	CM3 - Point Blinding (PB) 
	d.	CM4 – Scalar Splitting (SS)
\end{comment}

\subsubsection{Scalar Randomization (SR)}
\erick[inline]{Copiar conteudo slide SAC or UB}

\subsubsection{Projective Coordinates (Re-)Randomization (CR)}
\erick[inline]{Copiar conteudo slide SAC or UB}

\subsubsection{Point Blinding (PB)}
\erick[inline]{Copiar conteudo slide UB}

\subsubsection{Scalar Splitting (SS)}
\erick[inline]{TODO}


\begin{comment} % === Ataques SCA contra ECC devem ser do tipo single-trace  ===
5.	Explicar porque no contexto de PKC (RSA e ECC) não fazem sentido ataques que envolvem mais de um trace, como p.ex., DPA.
\end{comment}


\begin{comment}  % === ESTRUTURA ORIGINAL ===
\subsubsection{Ataques baseados em templates}
\subsubsection{Ataques horizontais baseados em cross-correlation}
\subsubsection{Ataques horizontais não-supervisionados baseados em clustering}

\subsubsection{Aplicação de contramedidas em algoritmo esquerda para direita inseguro}
\subsubsection{Implementações de tempo constante}
\subsubsection{Implementações resistentes ao SPA}
\subsubsection{Impacto das contramedidas no desempenho}

\subsection{Eficácia de implementação de tempo constante}
\subsubsection{Outros métodos para inviabilizar ataques por tempo}
\end{comment}

\subsection{Ataque SPA à alg. Montgomery Ladder com SR}

\erick[inline]{Descrever idéia de online template attack (OTA)~\cite{BatinaChmielewski2014}}

\subsection{Ataque SPA à alg. ECSM atômico com SR}
\erick[inline]{Explicar como online template attack (OTA)~\cite{BatinaChmielewski2014} pode ser aplicado neste caso}

\subsection{Ataque template SPA à alg. Montgomery Ladder com SR + CRR}

\erick[inline]{Copy-and-paste de partes do meu paper no SAC 2016.}

\subsection{Ataque HCA à alg. Montgomery Ladder c/ SR + CRR}
\erick[inline]{Copy-and-paste meu texto sobre HCA.}

\subsection{Ataques template versus Ataques horizontais}
\erick[inline]{Ataques template versus HCA: vantagens e desvantagens de cada um.}

\noindent \textbf{Precondições e limitações dos ataques baseados em template}: Ataques baseados em template são os mais poderosos ataques do tipo SCA, segundo a teoria da informação~\cite{ChariRaoRohatgi2003}. No entanto, ataques baseados em template só podem ser realizados quando a contramedida SR não é aplicada ou quando esta pode ser desabilitada durante a fase de criação de templates (profiling), caso contrário os templates não podem ser criados. Uma outra limitação deste tipo de ataque é de que dispositivos diferentes, mesmo que sejam do mesmo modelo, mesmo lote, etc., têm imperfeições únicas resultantes do processo de fabricação as quais resultam em diferenças no consumo de potência e radiação eletromagnética. Tais diferenças podem ser grandes o suficiente de modo que os templates gerados a partir dos traces provenientes do dispositivo de profiling não sejam bons modelos do vazamento observado no dispositivo alvo do ataque, assim reduzindo a taxa de sucesso do ataque~\cite{ElaabidGuilley2012}.


\noindent \textbf{Aplicabilidade}. Até então estes ataques só foram demonstrados em CPUs embarcadas de 8, 16 e 32 bits, devido ao alto nível de SNR (Signal-to-Noise Ratio) que pode ser obtido na medição no consumo de potência e EM nestes dispositivos. Quando o SNR é baixo, além de haver pouco vazamento de dados (data-leakage) explorável do valor da chave ou valores intermediários derivados deste, o alinhamento dos subtraces torna-se também inviável, devido a inexistência de intervalos próximos da ocorrência da operação alvo em que as amostras tem valores idênticos ou semelhantes em todos os subtraces.




\section{Recuperação de chaves com erros em criptosistemas baseados no (EC)DLP}

\subsection{Definições}
\erick[inline]{utilizar definições paper SAC}

\noindent \textbf{Pontuação de confiança}. No valor de um bit recuperado por ataque SCA.

\noindent \textbf{Nível de confiança}.

\subsection{Algoritmo de Lange, Vrendendaal e Wakker~\cite{LangeVredendaalWakker2014}}

\erick[inline]{baseado no algoritmo de Pollard-Kangaroo. Utilizar texto da discussão por email com os autores.}

\subsection{Algoritmo de Gopalakrishnan, Theriault e Yao~\cite{GopalakrishnanTheriaultYao07}}

\erick[inline]{copiar texto do paper SAC. Baseado no paradigma time-memory trade-off}

\subsection{Aplicação do Algoritmo de Gopalakrishnan, Theriault e Yao~\cite{GopalakrishnanTheriaultYao07} à curva elíptica Curve25519}

\erick[inline]{Explicar as otimizações de desempenho que podem ser aplicadas à nível de aritmética na curva}



\section{Ferramentas}

\subsection{Ferramentas para verificação de código de alto nível em relação à execução em tempo constante}


\subsection{Ferramentas para verificação de tempo constante}

Os primeiros métodos para verificação formal de contramedidas para canais laterais de tempo foram construídos a partir de análise estática. \cite{MolnarPSW05} propõem métodos para detectar canais laterais de controle de fluxo e transformar código fonte em C para eliminar as vulnerabilidades, abrangendo ataques de tempo e tratamento de erros. \cite{CoppensVBS09} modificam um compilador para converter comandos condicionais de forma que o código \emph{Assembly} resultante não mais tenha comportamento no tempo dependente dos dados processados.
~\cite{LuxS11} propõem uma ferramenta para detectar potenciais canais laterais em implementações em Java de algoritmos criptográficos, baseados em anotações do programador. Em~\cite{KopfMO12}, os autores propõem um método novo baseado em limitantes superiores automaticamente derivados na quantidade de informação sobre a entrada que o adversário consegue extrair de um programa a partir da observação do comportamento da \emph{cache} durante a execução. \cite{DoychevKMR15} propõem métodos para calcular aproximações precisas da quantidade de informação vazada que podem ser observadas nos canais laterais a seguir: transições de estado na \emph{cache}, traços de acertos e erros, e tempo de execução. Os autores também sugerem provas formais de segurança para contramedidas
como pré-carga de endereços e padrão de acesso independente de dados.

Ainda considerando ataques de tempo, outras abordagens para verificação foram propostas recentemente. O trabalho~\cite{AlmeidaBPV13} considera as políticas em alto nível adotadas na implementação da biblioteca NaCl:
ausência de desvios condicionais e endereçamento de vetores dependentes de dados;
formalizam as políticas e propõem um método de verificação formal baseado em auto-composição,
demonstrando-o pela aplicação no código \emph{Assembly} otimizado de algumas funções da biblioteca.
\cite{Langley12} propõe um método de análise dinâmico baseado no módulo \texttt{memcheck} da ferramenta \emph{Valgrind}, que amplifica sua capacidade para
reconhecer dados não-inicializados na granularidade de \emph{bits}. A ferramenta \emph{FlowTracker}\footnote{\url{http://cuda.dcc.ufmg.br/flowtracker/}}~\cite{RodriguesPA16} foi proposta recentemente para verificar comportamento constante de código compilado pela análise estática de fluxo de informação na representação intermediária LLVM. As vantagens da ferramenta são a facilidade de descrição das interfaces e a baixa intrusão, já que nenhuma alteração é necessária no código. CT-Verif~\cite{AlmeidaBBDE16} é uma ferramenta seguindo uma abordagem similar que fornece garantias formais adicionais, mas exige alteração do código pelo verificador.

\subsection{Ferramentas para verificação de implementações contra ataques de potência}

%\subsubsection{Aplicação semi-automática de contramedidas}

%==================================================================================================
%% texto proposta projeto Intel-Fapesp "Secure execution of cryptographic algorithms" de 2014
%==================================================================================================

%Formal verification also has been applied to power analysis. Moss et al.~\cite{moss2011automatic} introduce an algorithm to  automate the process of application of masking countermeasures against DPA, at the assembly code level. 

%%% Formal verification of software implementations of cryptographic algorithms against power analysis has been studied.
Verificação formal de implementações em software de algoritmos criptográficos contra análise de potência é um assunto que tem sido pesquisado recentemente.
%
% Maggi et al.~\cite{maggi2013automated, Agosta:2013:CSC:2463209.2488833} propose a method based on data flow analysis, to be able to identify dependencies of each instruction on secret data. 
%
Por exemplo, Maggi et al.~\cite{maggi2013automated, Agosta2013} propuseram um método baseado em análise de fluxo de dados para identificar dependências entre as instruções executadas e dados secretos.
%
% The method is implemented into the LLVM compiler as a specialized pass that works at the intermediate representation level, thus it is architecture agnostic, supporting any of the architectures supported by LLVM. The tool automatically instantiates the essential masking countermeasures. 
%
O método é implementado em um compilador LLVM como uma passada especializada operando no nível de representação intermediária, portanto ela é agnóstica em arquitetura, suportando quaisquer das arquiteturas de computador suportadas por LLVM. A ferramenta automaticamente instancia contramedidas de masking à implementação.
%

% Most recently, Bayrak et al~\cite{BayrakRegazzoniNovo:2013, Bayrak2014} showed how to reduce the verification problem, in the case of power channels, into a set of Boolean satisfiability problems, which can be efficiently handled by current SAT solvers, and which overcome some of shortcomings of the information flow analysis approach. 

Mais recentemente, Bayrak et al~\cite{BayrakRegazzoniNovo2013, Bayrak2014} mostraram como reduzir o problema de verificação da resistência de uma implementação a vazamento por canais de potência à um conjunto de problemas SAT, os quais podem ser eficientemente tratados pelos resolvedores SAT atuais. Tal método a princípio endereça as limitações da abordagem baseada em análise de fluxo de informação.

% Their approach motivated the introduction of satisfiability module theories (SMT) solvers by Eldib et al~\cite{EldibWang2014, EldibWang2014_QMS, EldibWang2014_SMT, EldibWang2014_sc_sniffer} to tackle the problem in the case of power analysis. The latter authors also proposed methods for the automated application of countermeasures.
%
Resolvedores baseados em teorias do módulo de satisfabilidade (SMT) foram aplicados por Eldib et al~\cite{EldibWang2014, EldibWang2014_QMS, EldibWang2014_SMT, EldibWang2014_sc_sniffer} a este problema. Estes últimos também propuseram métodos para a aplicação automatizada de contramedidas.



\subsection{Métodos empíricos para análise de vazamentos}

\subsubsection{Test Vector Leakage Assessment (TVLA)}

\erick[inline]{usar paper e slides SPACE e slides Riscure}



\subsection{Ferramentas para recuperação de chaves com erros}

\subsubsection{Criptossistemas baseados no (EC)DLP}
\erick[inline]{Citar a minha ferramenta de código aberto para a Curve25519}
\erick[inline]{Citar ferramenta do paper Kangaroos da Vrendevaal}

\subsubsection{AES}

\erick[inline]{Ferramentas de Veyrat~\cite{VeyratGerard2013,VeyratGerardStandaert2013} e da Vredendaal~\cite{BernsteinLangeVredendaal2015} para enumeração de chaves para o AES}


\bibliographystyle{sbc}
\bibliography{sbc-template}

%%%%%%%%%%%%%%%%%%%%% VERSÃO ENVIADA PARA O SBSEG %%%%%%%%%%%

\begin{comment} %%%%%%%%%%%%%%%%%%%%%% COMENTADA %%%%%%%%%%%%
\section{Dados Gerais}
\subsection{Objetivos do minicurso}
Curvas
O estudo de criptografia de curvas el\'ipticas (ECC) \'e muito importante nos dias atuais, já que possibilita o uso de chaves bem menores do que algoritmos como o RSA, mantendo o mesmo n\'ivel de segurança. Bibliotecas como OpenSSL, por exemplo, disponibilizam implementações de ECC para  assinatura digital (ECDSA) e troca de chaves com o esquema de Diffie-Hellman (ECDH).

Existem na literatura vários ataques de canal lateral a algoritmos de curvas el\'ipticas, o que obriga a implementa\c{c}\~ao de contramedidas de enrobustecimento, antes da entrada de tais métodos em produção. Tais ataques, em geral, visam a descoberta de chaves criptográficas, tendo como fonte de informação as diferenças de tratamento dado pelo código aos bits da chave: um desvio condicional ao valor de um bit pode levar a diferentes comportamentos que, por sua vez, produzem diferentes vazamentos de informação, seja no tempo de execução, consumo de potência, etc. 

Uma alternativa eficaz para mitigar esses problemas \'e a utiliza\c{c}\~ao de ferramentas para automatizar a inser\c{c}\~ao de contramedidas no c\'odigo. Essas ferramentas fazem a an\'alise do c\'odigo, podendo modificá-lo, ou tão somente informar onde se encontra a vulnerabilidade.

Considerando o  exposto acima, os objetivos deste minicurso s\~ao: discutir, brevemente, a necessidade e o impacto na segurança dos sistemas de IoT, do uso dos métodos criptográficos em uso atualmente em outros contextos; introduzir algoritmos de curvas el\'ipticas e suas variantes quanto às diferentes formas de implementação; introduzir e discutir protocolos de IoT baseados em ECC; discutir vários tipos de ataques de canal lateral, suas contramedidas, e limita\c{c}\~oes; descrever ferramentas para o fortalecimento de  algoritmos contra ataques de canal lateral.

\subsection{Tratamento dado ao tema}
Iremos abordar no minicurso aspectos te\'oricos de ataques de canal lateral aplicados a curvas el\'ipticas, e suas contramedidas. Faremos uma compara\c{c}\~ao entre ataques, destacando seu alcance e suas limita\c{c}\~oes. Do lado prático, vamos apresentar algumas  ferramentas que tornam o c\'odigo resistente a esse tipo de ataque. 

\subsection{Perfil desejado de audi\^encia}
O perfil desejado para a audiência do minicurso é o de estudantes e profissionais de computação e áreas correlatas interessados em segurança da informação e criptografia. Em especial, profissionais que se dediquem a implementar métodos criptográficos, sejam experientes ou não, deverão se beneficiar deste minicurso.  

\section{Estrutura do Minicurso}
\begin{enumerate}
\item \textbf{Introdução (3 pgs).} Seguran\c{c}a da informa\c{c}\~ao; Criptografia; Seguran\c{c}a em IoT.
\item \textbf {Encripta\c{c}\~ao sim\'etrica e hash (5 pgs).} AES~\cite{AES}; SHA-2~\cite{SHA2}; SHA-3~\cite{SHA3}.
\item \textbf{Criptografia de curvas el\'ipticas (ECC) (8 pgs).} 
    \begin{enumerate}
        \item Curvas NIST e curvas modernas;
        \item Algoritmos para multiplica\c{c}\~ao escalar (ECSM) b\'asicos~\cite{ECCBook_HankersonVanstone2004};
        \item Uso de tabelas precomputadas para melhorar desempenho~\cite{ECCBook_HankersonVanstone2004}; algoritmo de janela fixa;
        \item Algoritmos regulares~\cite{ECCBook_CohenFreyAvanzi2010}; atomicidade; Montgomery Ladder;
        \item Protocolos em IoT que envolvem ECC.
    \end{enumerate}
\item \textbf{Ataques por canais laterais (6 pgs).}
    \begin{enumerate}
        \item Canal lateral (SCA) de tempo~\cite{Kocher96};
            \begin{enumerate}
                \item O que \'e? Por que \'e importante?
                \item Limita\c{c}\~oes;
                \item Impacto em IoT;
            \end{enumerate}
        \item Canal lateral de pot\^encia~\cite{SCABook_Mangard2007};
        \begin{enumerate}
            \item Caracter\'isticas gerais;
            \begin{enumerate}
                \item Compara\c{c}\~ao entre canais laterais de pot\^encia e de tempo;
            \end{enumerate}
            \item Ataque de pot\^encia simples (SPA);
            \item Ataque de pot\^encia diferencial (DPA);
            \item High-order DPA;
            \item Template attacks.
        \end{enumerate}
    \end{enumerate}
\item \textbf{Ataques e contramedidas para encripta\c{c}\~ao sim\'etrica e hash (8 pgs).}
    \begin{enumerate}
        \item Utiliza\c{c}\~ao de CPA em implementa\c{c}\~ao do AES~\cite{SCABook_Mangard2007};
        \item Contramedida de mascaramento~\cite{SCABook_Mangard2007};
        \item Ataques e contramedidas para MACs baseados em funções de hash~\cite{McEvoyTunstall07,BertoniDaemenAssche09}.
    \end{enumerate}
\item \textbf{Ataques e contramedidas para ECC (20 pgs).}
    \begin{enumerate}
        \item N\'iveis em que os ataques SCA podem ser aplicados;
        \begin{enumerate}
            \item Opera\c{c}\~oes na curva;
            \item Opera\c{c}\~oes no protocolo criptogr\'afico;
            \item Transfer\^encia da chave entre diferentes mem\'orias;
            \item Protocolo em n\'ivel de aplica\c{c}\~ao;
        \end{enumerate}
        \item Ataques do tipo SPA e contramedidas~\cite{danger2013synthesis};
        \begin{enumerate}
            \item Ataque SPA clássico;
            \item Ataques baseados em templates~\cite{medwed2008template};
            \item Ataques horizontais baseados em cross-correlation~\cite{witteman2011defeating};
            \item Ataques horizontais não-supervisionados baseados em clustering~\cite{heyszl2013clustering, perin2015semi};
            \item Aplica\c{c}\~ao de contramedidas em algoritmo esquerda para direita inseguro;
            \item Implementa\c{c}\~oes de tempo constante;
            \item Implementa\c{c}\~oes resistentes ao SPA;
            \item Impacto das contramedidas no desempenho~\cite{coron1999resistance,danger2013synthesis};
        \end{enumerate}
        \item Efic\'acia de implementa\c{c}\~ao de tempo constante;
        \begin{enumerate}
            \item Outros métodos para inviabilizar ataques por tempo;
        \end{enumerate}
        \item Recuperação de chaves com erros; ~\cite{VeyratCharvillon2013,LangeVrendendaalWakker2014}
        \begin{enumerate}
            \item Forma de pontua\c{c}\~ao de confiança de um SCA;
            \item Pontua\c{c}\~ao em implementa\c{c}\~oes protegidas;
            \item Solu\c{c}\~ao para pontua\c{c}\~oes baixas de confiabilidade.
        \end{enumerate}
    \end{enumerate}
\item \textbf{Ferramentas (10 pgs).}
    \begin{enumerate}
        \item Ferramentas para verifica\c{c}\~ao de c\'odigo de alto n\'ivel em  relação \`a execu\c{c}\~ao em tempo constante~\cite{Langley2012};
        \item Ferramentas para an\'alise de c\'odigo assembly em relação à  prote\c{c}\~ao contra ataques de pot\^encia;
            \begin{enumerate}
                \item Aplica\c{c}\~ao semi-autom\'atica de contramedidas~\cite{MossOswald2012};
            \end{enumerate}
        \item M\'etodos emp\'iricos para an\'alise de vazamentos;
            \begin{enumerate}
                \item Test Vector Leakage Assessment (TVLA) e suas limita\c{c}\~oes~\cite{GoodwillJun2011, WittemanJaffe2013, TunstallGoodwill2016};
            \end{enumerate}
        \item Ferramentas para recuperação de chaves com erros~\cite{VeyratCharvillon2013,LangeVrendendaalWakker2014}.
    \end{enumerate}
\end{enumerate}

\section{Detalhamento da Estrutura}
\begin{enumerate}
\item \textbf{Introdução.} Introdu\c{c}\~ao à \'area de seguran\c{c}a, criptografia sim\'etrica e assim\'etrica, e a seguran\c{c}a na Internet das Coisas (IoT)

\item \textbf{Encripta\c{c}\~ao sim\'etrica e hash.} Apresenta\c{c}\~ao da encripta\c{c}\~ao sim\'etrica e hash, mas com um breve tratamento sobre o assunto. Nesse contexto ser\~ao apresentados um algoritmo sim\'etrico (AES) e duas funções de hash (SHA-2, SHA-3).

\item \textbf{Criptografia de curvas el\'ipticas (ECC).} Introdu\c{c}\~ao \`a matem\'atica utilizada em criptografia de curvas el\'ipticas. Ser\~ao apresentadas curvas padronizadas pelo NIST. Al\'em disso, diferentes formas de implementar curvas el\'ipticas, visando melhor desempenho, serão discutidas. E, por fim, protocolos de curvas el\'ipticas utilizados em IoT.

\item \textbf{Ataques por canais laterais.} Ser\~ao apresentados alguns ataques de canais laterais, como de tempo e de pot\^encia. No caso do de tempo \'e necess\'ario avaliar o impacto na utiliza\c{c}\~ao em aplica\c{c}\~oes de IoT; al\'em disso, expor as limita\c{c}\~oes desses ataques. J\'a os ataques de pot\^encia podem diferir  dependendo de como analisam o perfil de energia, podendo ser simples (SPA), diferenciais (DPA), ou de ordem mais alta (High-order DPA).

\item \textbf{Ataques e contramedidas para encripta\c{c}\~ao sim\'etrica e hash.} Nesse item iremos explorar a utiliza\c{c}\~ao de ataques de canal lateral em algoritmos que n\~ao s\~ao de ECC, como por exemplo AES. Isso mostra que um mesmo ataque pode ser utilizado em algoritmos diferentes, e at\'e mesmo baseados em problemas matem\'aticos diferentes. Al\'em disso, ser\'a apresentada uma contramedida ao ataque explorado.

\item \textbf{Ataques e contramedidas para ECC.} Ser\~ao apresentados ataques e suas contramedidas em algoritmos ECC. Isso ser\'a feito considerando o problema matem\'atico utilizado em curvas el\'ipticas. Para que seja poss\'ivel ter uma vis\~ao mais abrangente das contramedidas \'e necess\'ario avaliar o custo de mem\'oria e desempenho da aplica\c{c}\~ao das mesmas. Dependendo do sistema que utiliza algoritmos criptogr\'aficos, como sensores, pode n\~ao ser poss\'ivel fazer o uso de determinada contramedida. E, por fim, ser\'a apresentada a forma de pontua\c{c}\~ao dos bits da chave obtidos pela an\'alise de tempo.

\item \textbf{Ferramentas.} Para facilitar a implementa\c{c}\~ao de um c\'odigo seguro \'e poss\'ivel utilizar ferramentas que analisam o algoritmo e o tornam seguro, ou apenas apresentam as vulnerabilidades. Tamb\'em ser\~ao apresentados os m\'etodos para an\'alise do algoritmo que essas ferramentas podem utilizar.

\end{enumerate}

\bibliographystyle{sbc}
\bibliography{sbc-template}

\pagebreak 

\section{Curricula Vitae}
\subsection{Lucas Zanco Ladeira}
\subsubsection{Profissional}
2013 - 2014: Boa Vista Servi\c{c}os (Estagi\'ario).\\
Trabalhei com desenvolvimento web na linguagem Java (JEE). Foram utilizados frameworks como Spring, Struts 2, e Hibernate. Outras bibliotecas para facilitar o desenvolvimento foram utilizadas como JSTL. No projeto foi utilizado o motor de regras guvnor drools e servidores de aplica\c{c}\~ao JBoss e Tomcat.

\subsubsection{Educa\c{c}\~ao}
\begin{itemize}
\item 2014 - 2015: Bolsa de inicia\c{c}\~ao cient\'ifica FAPESP\\
A partir do meio do ano de 2014 fui contemplado com uma bolsa de inicia\c{c}\~ao cient\'ifica da FAPESP, para desenvolvimento de uma aplica\c{c}\~ao offline e segura de meio de pagamento com java card.\\
Durante esse trabalho foram pesquisados algoritmos criptogr\'aficos diferentes (AES, RSA, Esquema de Coron de Criptografia Homom\'orfica, Hash Sha-2) e t\'ecnicas de cria\c{c}\~ao de sess\~ao segura. Al\'em disso, formas diferentes de autentica\c{c}\~ao como PIN, biometria digital, e \textit{challenge-response authentication}.

\item 2016: Bolsa de mestrado da FAPESP\\
O projeto de mestrado tem o intuito de estudar algoritmos criptogr\'aficos, e suas implementa\c{c}\~oes seguras contra ataques de canal lateral. Isso torna necess\'ario o estudo dos diferentes ataques, e as contramedidas eficientes sobre cada uma. Al\'em disso, cada contramedida apresenta um n\'ivel de dificuldade diferente para implementa\c{c}\~ao, custo de processamento e mem\'oria.

\end{itemize}

\subsubsection{Publica\c{c}\~oes}
Durante o desenvolvimento da inicia\c{c}\~ao cient\'ifica as seguintes publica\c{c}\~oes foram conquistadas:\\
\begin{itemize}
\item Mecanismos de Autenticação em Smart Card utilizando Criptografia Totalmente Homomórfica. XXII Iberchip Workshop, 2016.
\item Autenticação em Java Card com Criptografia Homomórfica, V Congresso de Pesquisa Científica: Inovação, Sustentabilidade, Ética e Cidadania, 2015.
\item Meio de Pagamento Seguro e Off-line Utilizando Tecnologia Java Card. VII Congresso de Iniciação em Desenvolvimento Tecnológico e Inovação da UFSCar, 2014.
\end{itemize}

\subsection{Erick Nogueira do Nascimento}

\subsubsection*{Formação}
\begin{itemize}\setlength\itemsep{1pt}
\item 2004 a 2008, Bacharelado em Engenharia de Computação, UNICAMP.
\item 2009 a 2011, Mestrado em Ciência da Computação, UNICAMP.
\item 2011 (cursando), Doutorado em Ciência da Computação, UNICAMP.
\end{itemize}

\subsubsection*{Histórico Profissional}

É candidato ao doutorado em Ciência da Computação na UNICAMP. Atualmente realiza um estágio de pesquisa na empresa Riscure BV nos Países Baixos, com o tema de ataques por canais laterais de potência e radiação eletromagnética contra implementações em software de algoritmos criptográficos assimétricos. Foi bolsista do Ciência sem Fronteiras por um ano, de Abril de 2015 a Abril de 2016, modalidade de estágio de doutorado sanduíche no exterior, na Radboud University Nijmegen, Países Baixos, sob orientação do Prof. Peter Schwabe. Foi pesquisador no CPqD de 2012 a 2014, onde atuou principalmente em projetos envolvendo criptografia em software e hardware, bem como engenharia reversa de software, análise de vulnerabilidades e testes de intrusão em aplicações. 

\subsubsection*{Publicações Recentes}

\begin{enumerate}\setlength\itemsep{1pt}
    \item Nascimento, E., López, J. and Dahab, R., "Efficient and secure elliptic curve cryptography for 8-bit avr microcontrollers." \textsl{International Conference on Security, Privacy, and Applied Cryptography Engineering (SPACE)}. Springer International Publishing, 2015. 
    \item Kawakami, H., Gallo, R., Dahab, R., \& Nascimento, E., “Hardware Security Evaluation Using Assurance Case Models”. In \textsl{Availability, Reliability and Security (ARES), 2015 10th International Conference on} (pp. 193-198). IEEE, 2015.
    \item Nascimento, E., Abarzua, R., Lopez, J., Dahab, R., “A comparison of simple side-channel analysis countermeasures for variable-base elliptic curve scalar multiplication”. \textsl{Proceedings of the Brazilian Symposium on Information and Systems Security}, 2014.
\end{enumerate}

\subsection{João Paulo Fernandes Ventura}

\subsubsection{Formação}
\begin{itemize}\setlength\itemsep{1pt}
\item 2003 a 2007, Graduação em Engenharia de Computação, UNICAMP.
\end{itemize}

\subsubsection{Histórico Profissional}
\begin{itemize}\setlength\itemsep{1pt}
\item 2010 à 2012, \textit{Software Engineer} na IBM, atuando no desenvolvimento de virtual appliances para \textit{RHEV Blue}.
\item 2013 à 2013, \textit{Software Engineer} na Intel Corporation, atuando no desenvolvimento do Tizen OS.
\item 2013 à 2014, \textit{Software Developer} no Samsung Instituto de Desenvolvimento para a Informática, atuando no desenvolvimento do Samsung Knox.
\end{itemize}

\subsubsection{Publicações Recentes}
\begin{itemize}\setlength\itemsep{1pt}
\item Ventura, J.P.F., Dahab, R., "Introduction to Side-Channel Attacks". In \textit{IX Brazilian Symposium on Information and Computational Systems Security}, 2009.
\end{itemize}

\subsection{Ricardo Dahab}
\subsubsection{Formação}
\begin{itemize}
\item 1978 - Bacharelado em Ciência da Computação, UNICAMP
\item 1984 - Mestrado em Ciência da Computação, UNICAMP
\item 1993 - Ph.D. em Combinatória e Otimização, University of Waterloo, Canadá
\item 2002 - Livre-docência em Complexidade de Algoritmos, UNICAMP
\end{itemize}
\subsubsection{Histórico Profissional}
Professor em tempo integral na UNICAMP desde 1982, com promoções em 1984 para professor assistente, depois em 1993 para assistente-doutor, e em 2002 para professor associado. 

\subsubsection{Publicações Recentes}
Link para CV Lattes: \texttt{http://lattes.cnpq.br/9093331241572944}
\begin{enumerate}\setlength\itemsep{1pt}
    \item Nascimento, E., López, J. and Dahab, R., "Efficient and secure elliptic curve cryptography for 8-bit avr microcontrollers." \textsl{International Conference on Security, Privacy, and Applied Cryptography Engineering (SPACE)}. Springer International Publishing, 2015. 
    \item Kawakami, H., Gallo, R., Dahab, R., \& Nascimento, E., “Hardware Security Evaluation Using Assurance Case Models”. In \textsl{Availability, Reliability and Security (ARES), 2015 10th International Conference on} (pp. 193-198). IEEE, 2015.
    \item Nascimento, E., Abarzua, R., Lopez, J., Dahab, R., “A comparison of simple side-channel analysis countermeasures for variable-base elliptic curve scalar multiplication”. \textsl{Proceedings of the Brazilian Symposium on Information and Systems Security}, 2014.
\end{enumerate}

\subsection{Diego F. Aranha}

\subsubsection*{Formação}
\begin{itemize}\setlength\itemsep{1pt}
\item 2000 a 2005, Graduação em Ciência da Computação, Universidade de Brasília.
\item 2005 a 2007, Mestrado Ciência da Computação, UNICAMP.
\item 2007 a 2011, Doutorado em Ciência da Computação pela UNICAMP.
\end{itemize}

\subsubsection*{Histórico Profissional}

É Professor Doutor na Universidade Estadual de Campinas (Unicamp) desde 2014. Tem experiência na área de Criptografia e Segurança Computacional, com ênfase em implementação eficiente de algoritmos criptográficos e análise de segurança de sistemas reais. Coordenou a primeira equipe de investigadores independentes capaz de detectar e explorar vulnerabilidades no software da urna eletrônica em testes controlados organizados pelo Tribunal Superior Eleitoral. É Bacharel em Ciência da Computação pela Universidade de Brasília (2005), Mestre (2007) e Doutor (2011) em Ciência da Computação pela Universidade Estadual de Campinas. Foi doutorando visitante por 1 ano na Universidade de Waterloo, Canadá, e Professor Adjunto por pouco mais de 2 anos no Departamento de Ciência da Computação da Universidade de Brasília. É membro do Comitê Consultivo da Conferência Internacional em Criptografia e Segurança da Informação na América Latina (LATINCRYPT) e da Comissão Especial de Segurança da Sociedade Brasileira de Computação (CESEG), responsável pelo Simpósio Brasileiro de Segurança da Informação e Sistemas Computacionais (SBSEG), tendo coordenado o Comitê de Programa na edição 2014 de ambos os eventos. Recebeu em 2015 os prêmios Google Latin America Research Award para pesquisa em privacidade e Inovadores com Menos de 35 Anos Brasil da MIT Technology Review por seu trabalho com o voto eletrônico.

\subsubsection*{Publicações relevantes}

\begin{enumerate}\setlength\itemsep{1pt}
    \item T. Oliveira, J. López, {D. F. Aranha}, F. Rodr\'iguez-Henr\'iquez, ``Lambda coordinates for binary elliptic curves'',
    In \textsl{15th International Workshop on Cryptographic Hardware and Embedded Systems (CHES 2013)}, Springer LNCS 8086, pp. 311--330, Santa Barbara, USA, 2013. \textbf{Best Paper Award!}

    \item L. B. Oliveira, {D. F. Aranha}, C. P. L. Gouvêa, M. Scott, D. F. Câmara, J. López, R. Dahab.
    ``TinyPBC: Pairings for Authenticated Identity-Based Non-Interactive Key Distribution in Sensor Networks'', 
    \textsl{Computer Communications}, Vol. 34, Issue 3, pp. 485--493, 2011.

    \item {D. F. Aranha}, K. Karabina, P. Longa, C. H. Gebotys, J. López.
    ``Faster Explicit Formulas for Computing Pairings over Ordinary Curves'',
    In \textsl{30th International Conference on the Theory and Applications of Cryptographic Techniques (EUROCRYPT 2011)}, Springer LNCS 6632, pp. 48--68, Tallinn, Estonia, 2011.
\end{enumerate}

\subsection{Julio L\'opez}

\subsubsection*{Formação}
\begin{itemize}\setlength\itemsep{1pt}
\item 1979 a 1983, Graduação em Matemática, Universidad del Valle.
\item 1983 a 1985, Mestrado  em Matemática Aplicada, Universidad del Valle.
\item 1995 a 2000, Doutorado em Ciência da Computação no Instituto da Computação da UNICAMP com período sanduíche em University of Waterloo.
\item 2009, Livre-docência pela UNICAMP.
\end{itemize}

\subsubsection*{Histórico profissional}

Professor Associado no Instituto de Computação da Universidade Estadual de Campinas. Tem experiência na área de Ciência da Computação, com ênfase em Engenharia Criptográfica, atuando principalmente nos seguintes temas: algoritmos criptográficos, implementação hardware/software de criptossistemas de curvas elípticas, bibliotecas criptográficas, aritmética computacional e aplicações criptográficas.

\subsubsection*{Publicações relevantes}
\begin{enumerate}\setlength\itemsep{1pt}
\item Danilo Câmara, Conrado PL Gouvêa, Julio López, and Ricardo Dahab. \emph{Fast software polynomial multiplication on arm processors using the neon engine}. In Security Engineering and Intelligence Informatics, pages 137-154. Springer, 2013.

\item D. F. Aranha, J. López, and D. Hankerson. \emph{Efficient Software Implementation of Binary Field Arithmetic Using Vector Instruction Sets}. In M. Abdalla and P. S. L. M. Barreto, editors, The First International Conference on Cryptology and Information Security (LATINCRYPT 2010), volume 6212 of LNCS, pages 144-161, 2010.

\item Conrado PL Gouvêa, Leonardo B Oliveira, and Julio López. \emph{Efficient software implementation of public-key cryptography on sensor networks using the msp430x microcontroller}. Journal of Cryptographic Engineering, 2(1):19-29, 2012.
\item C. P. L. Gouv\^ea and J. L\'opez. Implementing GCM on armv8. In Topics in Cryptology - CT-RSA 2015, The Cryptographer's Track at the RSA Conference 2015, San Francisco, CA, USA, April 20-24, 2015. Proceedings, pages 167,180, 2015.
\end{enumerate}

\end{comment}

\end{document}
