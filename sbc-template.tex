\documentclass[12pt]{article}

\usepackage{sbc-template}
\usepackage{verbatim}
\setcounter{secnumdepth}{4}
\usepackage{graphicx,url}

\usepackage{amsmath}
\usepackage{amsfonts}
\usepackage{algorithm}
\usepackage{algorithmic}
\usepackage{multicol}


% Algorithmic modifications
\makeatletter
\newcommand{\ALOOP}[1]{\ALC@it\algorithmicloop\ #1%
  \begin{ALC@loop}}
\newcommand{\ENDALOOP}{\end{ALC@loop}\ALC@it\algorithmicendloop}
\renewcommand{\algorithmicrequire}{\textbf{Entrada:}}
\renewcommand{\algorithmicensure}{\textbf{Sa\'ida:}}
\newcommand{\algorithmicbreak}{\textbf{break}}
\newcommand{\BREAK}{\STATE \algorithmicbreak}
\makeatother

\usepackage[brazil]{babel}   
%\usepackage[latin1]{inputenc}  
\usepackage[utf8]{inputenc}  
% UTF-8 encoding is recommended by ShareLaTex


\sloppy

%\title{Canais laterais em criptografia simétrica e de curvas elípticas: ataques e contramedidas\footnote{Esse trabalho é uma versão aprofundada e modernizada do minicurso ministrado no SBSeg 2009 por João Paulo Fernandes Ventura e Ricardo Dahab}}

\title{Canais laterais em criptografia simétrica e de curvas elípticas: ataques e contramedidas}

\author{Lucas Zanco Ladeira\inst{1} (apresentador), Erick Nascimento\inst{1}, Jo\~ao Ventura\inst{1}\and \\ Ricardo Dahab\inst{1}, Diego Aranha\inst{1}, Julio Lopez\inst{1}}


\address{Instituto de Computação – Universidade Estadual de Campinas (Unicamp)\\
Caixa Postal 6176 – 13.084-970 – Campinas – SP – Brasil
    \email{lucas.ladeira@students.ic.unicamp.br}
  \email{\{erick.nogueira.nascimento,joao.ventura\}@gmail.com}
  \email{\{rdahab,dfaranha,jlopez\}@ic.unicamp.br}
}

\begin{document} 

\maketitle
     
\begin{resumo}
Existe uma infinidade de ataques visando a quebra de sistemas criptogr\'aficos, buscando a obtenção de informação sigilosa, acesso não-autorizado, entre outros. Tais ataques ocorrem tanto sobre os algoritmos quanto implementações de um sistema. Um tipo de ataque sobre implementações \'e o de canais laterais, que faz uso do vazamento de informa\c{c}\~oes durante a execu\c{c}\~ao de alguma primitiva criptogr\'afica. Ataques dessa categoria utilizam variações do tempo de execução, do consumo de pot\^encia, do campo magn\'etico, entre outros. As contramedidas podem ser baseadas em modificações no software ou no hardware. Neste minicurso, nossa atenção se restringe ao software, voltada para métodos criptográficos simétricos, e os assimétricos baseados em curvas elípticas.
\end{resumo}

\pagebreak
\section{Introdução}

\pagebreak
\section{Encriptação simétrica e hash}

\pagebreak
\section{Criptografia de curvas elípticas (ECC)}
Criptografia de curvas elípticas é uma classe de algoritmos criptográficos que se baseia na aritmética de pontos de uma curva elíptica em um corpo finito $\mathbb{F}$ \cite{Hankerson:2003:GEC:940321}. Os algoritmos criptográficos que utilizam esse tipo de problema matemático podem se diferenciar de acordo com vários fatores, como por exemplo: o primo utilizado, a curva, corpo finito $\mathbb{F}_p$ ou $\mathbb{F}_{2^m}$, o mapeamento dos pontos na curva, entre outros. 

As operações mais simples executadas em uma curva é a adição de pontos e a duplicação de um ponto, sendo $R = P + Q$ e $R = P + P$ respectivamente. Dependendo da curva utilizada é possível executar essas operações de maneira mais eficiente podendo diferenciar o mapeamento desses pontos na curva.

Seguindo a notação descrita por \cite{Hankerson:2003:GEC:940321} podemos definir uma curva elíptica $E$ como: $E: y^2 + a_1xy + a_3y = x^3 + a_2x^2 + a_4x + a_6$ sobre um corpo K. Onde $\bigtriangleup \neq 0$, essa condição garante que não existirá ponto onde a curva possui duas ou mais linhas de tangente diferentes. A linha da tangente é utilizada na operação de duplicação de um ponto. Sendo que, $\bigtriangleup$ é definido como:

\begin{align*}
\bigtriangleup &= -d_2^2d_8 - 8d_4^3 - 27d_6^2 + 9d_2d_4d_6 \\
d_2 &= a_1^2 + 4a_2 \\
d_4 &= 2a_4 + a_1a_3 \\ 
d_6 &= a_3^2 + 4a_6 \\
d_8 &= a_1^2a_6 + 4a_2a_6 - a_1a_3a_4 + a_2a_3^2 - a_4^2
\end{align*}

A curva descrita anteriormente é chamada de equação de Weiertrass, onde a mesma possui uma forma simplificada: $y^2 = x^3 + ax + b$. Essa equação é utilizada como ponto inicial para descrição de várias curvas, sendo que pode-se variar os valores de $a$ e $b$ para obter diferentes curvas. Durante a escolha de qual utilizar é possível verificar a facilidade de implementação, possibilidade de paralelização, e também o desempenho da mesma. 

O desempenho está ligado as operações executadas, como por exemplo o método Montgomery-Ladder utilizado durante a multiplicação escalar, e pelas instruções do processador que facilitam a aritmética. Na seção a seguir iremos exibir as curvas com mais ocorrência na literatura, e algumas propriedades de cada uma.

\subsection{Curvas NIST e curvas modernas}
Primeiramente iremos apresentar as curvas NIST, essas curvas são recomendações de utilização feitas pelo NIST (National Institute of Standards and Technlogy), situado nos Estados Unidos da América. As mesmas foram geradas de forma pseudo-aleatória pela NSA, e possuem ao todo são 10 corpos finitos sendo 5 corpos primos ($\mathbb{F}_p$) e 5 corpos binários ($\mathbb{F}_{2^m}$) \cite{Brown2001}.

Os corpos foram recomendados com o foco no desempenho das curvas, facilitando a aritmética utilizada. Todavia existe uma resistência da comunidade em adotar o que foi proposto, pela incerteza na existência de vulnerabilidades, inseridas para obter informações secretas pelo governo norte americano. Os corpos finitos recomendados podem ser observados a seguir, sendo P os corpos primos e B as corpos binários:

\begin{multicols}{2}
\begin{itemize}
\item P-192 \\ $\mathbb{F}_{192}$ $p = 2^{192} - 2^{64} - 1$; 
\item P-224 \\ $\mathbb{F}_{224}$ $p = 2^{224} - 2^{96} + 1$;
\item P-256 \\ $\mathbb{F}_{256}$ $p = 2^{256} - 2^{224} + 2^{192} + 2^{96} - 1$;
\item P-384 \\ $\mathbb{F}_{384}$ $p = 2^{384} - 2^{128} - 2^{96} + 2^{32} - 1$;
\item P-521 \\ $\mathbb{F}_{521}$ $p = 2^{521} - 1$;
\item B-163 \\ $\mathbb{F}_{2^{163}}$ $f(x) = x^{163} + x^7 + x^6 + x^3 + 1$;
\item B-233 \\ $\mathbb{F}_{2^{233}}$ $f(x) = x^{233} + x^{74} + 1$;
\item B-283 \\ $\mathbb{F}_{2^{283}}$ $f(x) = x^{283} + x^{12} + x^7 + x^5 + 1$;
\item B-409 \\ $\mathbb{F}_{2^{409}}$ $f(x) = x^{409} + x ^{87} + 1$;
\item B-571 \\ $\mathbb{F}_{2^{571}}$ $f(x) = x^{571} + x^{10} + x^5 + x^2 + 1$.
\end{itemize}
\end{multicols}

Nesses corpos finitos primos é recomendado utilizar as curvas pseudo-aleatórias, já no caso dos corpos binários recomenda-se, além da utilização das curvas pseudo-aleatórias, o uso da curva de Koblitz. A geração de curvas pseudo-aleatórias segue três passos: primeiramente é gerada uma semente, após a partir da semente é gerada uma curva. Por fim, é verificado se a curva gerada é resistente aos ataques conhecidos, caso não seja o processo é repetido.

Existem duas curvas elípticas que estão se destacando considerando o desempenho obtido junto à técnicas de multiplicação escalar eficiente, redução modular, entre outras. Elas são a curva de Montgomery e a curva de Edwards, onde é possível encontrar implementações resistentes a ataques de canal lateral como ataque por tempo e ataque por cache.

A curva de Montgomery tem a seguinte equação: $E: y^2 = x^3 + Ax^2 + x$. O valor do parâmetro $A$ pode ser alterado para melhorar o desempenho das multiplicações escalares. Iremos tratar a utilização dessa curva com o primo $25519$ $(2^{255}-19)$ e o valor de $A = 486662$, dessa maneira a curva é definida sobre o corpo $\mathbb{F}_{2^{255}-19}$ \cite{Dull:2015:HCM:2834659.2834708}. Considerando o corpo finito utilizado essa curva é chamada de 25519, onde os pontos na curva são mapeados como $P = (X : Z)$. 

A curva de Edwards tem a fórmula: $E: -x^2 + y^2 = 1 + dx^2y^2$ \cite{Bernstein2012}. Nessa curva a soma de dois pontos segue a \textit{Edwards addition law}:
$$ (x_1,y_1) + (x_2,y_2) = (\frac{x_1y_2 + x_2y_1}{1 + xd_1x_2y_1y_2},\frac{y_1y_2 + x_1x_2}{1 - dx_1x_2y_1y_2}) $$%//TODO

Uma contramedida encontrada de fácil implementação para se tornar resistente a alguns ataques de canal lateral, como ataque de cache e hyperthreading, é possível apenas não escrever índices de dados secretos na memória RAM. Considerando que os ataques mencionados dependem das informações obtidas pela utilização dessas memórias eles não seriam efetivos. 

Um outro ataque conhecido é o ataque por tempo, o mesmo analisa o tempo de execução da primitiva criptográfico que utiliza a chave secreta, para tentar obter os bits da chave. A resistência do algoritmo está baseada na execução em tempo constante, o que pode ser obtido sem a utilização de branchs condicionais. Na seção 6 do minicurso iremos apresentar melhor essa contramedida e a eficácia da mesma.

\subsection{Algoritmos para multiplicação escalar (ECSM) básicos}

% Left-to-right binário (a.k.a. Point and Add Not-Always).
% Left-to-right binário com contramedida “Dummy Adds” de Coron  (a.k.a. Point and Add Always).
% Atomic Point and Add. Apresentar a idéia geral, e talvez um conjunto de fórmulas para uma dada curva como exemplo.
% Montgomery Ladder. Apenas a versão para curvas na forma de Montgomery.

Existem vários métodos para executar a multiplicação escalar, sendo que os mais simples de implementar serão apresentados nessa seção. O primeiro é o método \textit{Double-and-add}, o mesmo é utilizado para calcular $dP$. Ele possui quatro entradas, sendo $N$ uma variável para armazenar o resultado da operação de duplicação do ponto, Q armazena o resultado das operações, P é o ponto escolhido na curva para a multiplicação escalar, e por fim $d$ é a chave utilizada na multiplicação. É importante citar que $m$ é o tamanho da chave em bits, sendo que cada bit da chave será utilizado nesse método, totalizando 256 execuções para uma chave de 32 bytes.

\floatname{algorithm}{Algoritmo}
\begin{algorithm}[H]
\caption{Double-and-add}
\begin{algorithmic} 
    \REQUIRE $N, Q, P, d$
    \ENSURE $Q$
    \STATE $N \leftarrow P$
    \STATE $Q \leftarrow 0$
    \FOR{$i$ from $0$ \TO $m$} 
        \IF{$d_i = 1$}
            \STATE $Q \leftarrow addpoint(Q, N)$
        \ENDIF
        \STATE $N \leftarrow doublepoint(N)$
    \ENDFOR
    \RETURN Q
    \end{algorithmic}
\end{algorithm}


\subsection{Uso de tabelas precomputadas para melhorar desempenho; algoritmos de janela fixa}

\subsection{Algoritmos regulares; atomicidade; montgomery ladder}

Utilizando a curva de Montgomery é possível executar a multiplicação escalar pelo método chamado Montgomery Ladder. Esse método é dividido em degraus, sendo necessário executar 255 vezes para concluir a computação \cite{Dull:2015:HCM:2834659.2834708}.

\begin{algorithm}[H]
\caption{Montgomery ladder}
\begin{algorithmic} 
    \REQUIRE $s, x_p$
    \ENSURE $X_1, Z_1$
    \STATE $X_1 \leftarrow 1$
    \STATE $Z_1 \leftarrow 0$
    \STATE $X_2 \leftarrow x_p$
    \STATE $Z_2 \leftarrow 1$
    \STATE $p \leftarrow 0$
    \FOR{$i \leftarrow 254$ \TO $0$}
        \STATE $b \leftarrow s_i$
        \STATE $c \leftarrow b \oplus p$
        \STATE $p \leftarrow b$
        \STATE $(X_1, X_2) \leftarrow cswap(X_1,X_2,c)$
        \STATE $(Z_1,Z_2) \leftarrow cswap(Z_1,Z_2,c)$
        \STATE $(X_1, Z_1, X_2, Z_2) \leftarrow ladder(x_p, X_1, Z_1, X_2, Z_2)$
    \ENDFOR
    \RETURN $(X_1,Z_1)$
    \end{algorithmic}
\end{algorithm}

\begin{algorithm}[H]
\caption{Ladder}
\begin{multicols}{2}
\begin{algorithmic} 
    \REQUIRE $x_D, X_1, Z_1, X_2, Z_2$
    \ENSURE $X_1, Z_1, X_2, Z_2$
    \STATE $T_1 \leftarrow X_2 + Z_2$
    \STATE $X_2 \leftarrow X_2 - Z_2$
    \STATE $Z_2 \leftarrow X_1 + Z_1$
    \STATE $X_1 \leftarrow X_1 - Z_1$
    \STATE $T_1 \leftarrow T_1 \cdot X_1$
    \STATE $X_2 \leftarrow X_2 \cdot Z_2$
    \STATE $Z_2 \leftarrow Z_2 \cdot Z_2$
    \STATE $X_1 \leftarrow X_1 \cdot X_1$
    \STATE $T_2 \leftarrow Z_2 - X_1$
    \STATE $Z_1 \leftarrow T_2 \cdot a_24$
    
    
    \STATE $Z_1 \leftarrow Z_1 + X_1$
    \STATE $Z_1 \leftarrow T_2 \cdot Z_1$
    \STATE $X_1 \leftarrow Z_2 \cdot X_1$
    \STATE $Z_2 \leftarrow Z_2 - X_2$
    \STATE $Z_2 \leftarrow Z_2 \cdot Z_2$
    \STATE $Z_2 \leftarrow Z_2 \cdot x_D$
    \STATE $X_2 \leftarrow T_1 + X_2$
    \STATE $X_2 \leftarrow X_2 \cdot X_2$
    
    \RETURN $(X_1,Z_1)$
\end{algorithmic}
\end{multicols}
\end{algorithm}

\subsection{Protocolos de IoT que envolvem ECC}

\pagebreak
\section{Ataques por canais laterais}

\subsection{Canal lateral (SCA) de tempo}

\subsubsection{O que é? Por que é importante?}

\subsubsection{Impacto em IoT}

\subsubsection{Limitações}

\subsection{Canal lateral de potência}

\subsubsection{Características gerais}

\paragraph{Comparação entre canais laterais de potência e de tempo}

\subsubsection{Ataque de potência simples (SPA)}

\subsubsection{Ataque de potência diferencial (DPA)}

\subsubsection{High-order DPA}

\subsubsection{Template attacks}

\pagebreak
\section{Ataques e contramedidas para encriptação simétrica e hash}

\subsection{Utilização de CPA em implementação do AES}

\subsection{Contramedida de mascaramento}

\subsection{Ataques e contramedidas para MACs baseados em funções de
hash}

\pagebreak
\section{Ataques e contramedidas para ECC}

\subsection{Níveis em que os ataques SCA podem ser aplicados}

\subsubsection{Operações na curva}

\subsubsection{Operações no protocolo criptográfico}

\subsubsection{Transferência da chave entre diferentes memórias}

\subsubsection{Protocolo em nível de aplicação}

\subsection{Ataques do tipo SPA e contramedidas}

\subsubsection{Ataque SPA clássico}

\subsubsection{Ataques baseados em templates}

\subsubsection{Ataques horizontais baseados em cross-correlation}

\subsubsection{Ataques horizontais não-supervisionados baseados em clustering}

\subsubsection{Aplicação de contramedidas em algoritmo esquerda para direita
inseguro}

\subsubsection{Implementações de tempo constante}

\subsubsection{Implementações resistentes ao SPA}

\subsubsection{Impacto das contramedidas no desempenho}

\subsection{Eficácia de implementação de tempo constante}

\subsubsection{Outros métodos para inviabilizar ataques por tempo}

\subsection{Recuperação de chaves com erros}

\subsubsection{Forma de pontuação de confiança de um SCA}

\subsubsection{Pontuação em implementações protegidas}

\subsubsection{Solução para pontuações baixas de confiabilidade}

\pagebreak
\section{Ferramentas}

\subsection{Ferramentas para verificação de código de alto nível em relação à execução em tempo constante}

\subsection{Ferramentas para análise de código assembly em relação à proteção contra ataques de potência}

\subsubsection{Aplicação semi-automática de contramedidas}

\subsection{Métodos empíricos para análise de vazamentos}

\subsubsection{Test Vector Leakage Assessment (TVLA) e suas limitações}

\subsection{Ferramentas para recuperação de chaves com erros}

\bibliographystyle{sbc}
\bibliography{sbc-template}

%%%%%%%%%%%%%%%%%%%%% VERSÃO ENVIADA PARA O SBSEG %%%%%%%%%%%

\begin{comment} %%%%%%%%%%%%%%%%%%%%%% COMENTADA %%%%%%%%%%%%
\section{Dados Gerais}
\subsection{Objetivos do minicurso}
Curvas
O estudo de criptografia de curvas el\'ipticas (ECC) \'e muito importante nos dias atuais, já que possibilita o uso de chaves bem menores do que algoritmos como o RSA, mantendo o mesmo n\'ivel de segurança. Bibliotecas como OpenSSL, por exemplo, disponibilizam implementações de ECC para  assinatura digital (ECDSA) e troca de chaves com o esquema de Diffie-Hellman (ECDH).

Existem na literatura vários ataques de canal lateral a algoritmos de curvas el\'ipticas, o que obriga a implementa\c{c}\~ao de contramedidas de enrobustecimento, antes da entrada de tais métodos em produção. Tais ataques, em geral, visam a descoberta de chaves criptográficas, tendo como fonte de informação as diferenças de tratamento dado pelo código aos bits da chave: um desvio condicional ao valor de um bit pode levar a diferentes comportamentos que, por sua vez, produzem diferentes vazamentos de informação, seja no tempo de execução, consumo de potência, etc. 

Uma alternativa eficaz para mitigar esses problemas \'e a utiliza\c{c}\~ao de ferramentas para automatizar a inser\c{c}\~ao de contramedidas no c\'odigo. Essas ferramentas fazem a an\'alise do c\'odigo, podendo modificá-lo, ou tão somente informar onde se encontra a vulnerabilidade.

Considerando o  exposto acima, os objetivos deste minicurso s\~ao: discutir, brevemente, a necessidade e o impacto na segurança dos sistemas de IoT, do uso dos métodos criptográficos em uso atualmente em outros contextos; introduzir algoritmos de curvas el\'ipticas e suas variantes quanto às diferentes formas de implementação; introduzir e discutir protocolos de IoT baseados em ECC; discutir vários tipos de ataques de canal lateral, suas contramedidas, e limita\c{c}\~oes; descrever ferramentas para o fortalecimento de  algoritmos contra ataques de canal lateral.

\subsection{Tratamento dado ao tema}
Iremos abordar no minicurso aspectos te\'oricos de ataques de canal lateral aplicados a curvas el\'ipticas, e suas contramedidas. Faremos uma compara\c{c}\~ao entre ataques, destacando seu alcance e suas limita\c{c}\~oes. Do lado prático, vamos apresentar algumas  ferramentas que tornam o c\'odigo resistente a esse tipo de ataque. 

\subsection{Perfil desejado de audi\^encia}
O perfil desejado para a audiência do minicurso é o de estudantes e profissionais de computação e áreas correlatas interessados em segurança da informação e criptografia. Em especial, profissionais que se dediquem a implementar métodos criptográficos, sejam experientes ou não, deverão se beneficiar deste minicurso.  

\section{Estrutura do Minicurso}
\begin{enumerate}
\item \textbf{Introdução (3 pgs).} Seguran\c{c}a da informa\c{c}\~ao; Criptografia; Seguran\c{c}a em IoT.
\item \textbf {Encripta\c{c}\~ao sim\'etrica e hash (5 pgs).} AES~\cite{AES}; SHA-2~\cite{SHA2}; SHA-3~\cite{SHA3}.
\item \textbf{Criptografia de curvas el\'ipticas (ECC) (8 pgs).} 
    \begin{enumerate}
        \item Curvas NIST e curvas modernas;
        \item Algoritmos para multiplica\c{c}\~ao escalar (ECSM) b\'asicos~\cite{ECCBook_HankersonVanstone2004};
        \item Uso de tabelas precomputadas para melhorar desempenho~\cite{ECCBook_HankersonVanstone2004}; algoritmo de janela fixa;
        \item Algoritmos regulares~\cite{ECCBook_CohenFreyAvanzi2010}; atomicidade; Montgomery Ladder;
        \item Protocolos em IoT que envolvem ECC.
    \end{enumerate}
\item \textbf{Ataques por canais laterais (6 pgs).}
    \begin{enumerate}
        \item Canal lateral (SCA) de tempo~\cite{Kocher96};
            \begin{enumerate}
                \item O que \'e? Por que \'e importante?
                \item Impacto em IoT;
                \item Limita\c{c}\~oes.
            \end{enumerate}
        \item Canal lateral de pot\^encia~\cite{SCABook_Mangard2007};
        \begin{enumerate}
            \item Caracter\'isticas gerais;
            \begin{enumerate}
                \item Compara\c{c}\~ao entre canais laterais de pot\^encia e de tempo;
            \end{enumerate}
            \item Ataque de pot\^encia simples (SPA);
            \item Ataque de pot\^encia diferencial (DPA);
            \item High-order DPA;
            \item Template attacks.
        \end{enumerate}
    \end{enumerate}
\item \textbf{Ataques e contramedidas para encripta\c{c}\~ao sim\'etrica e hash (8 pgs).}
    \begin{enumerate}
        \item Utiliza\c{c}\~ao de CPA em implementa\c{c}\~ao do AES~\cite{SCABook_Mangard2007};
        \item Contramedida de mascaramento~\cite{SCABook_Mangard2007};
        \item Ataques e contramedidas para MACs baseados em funções de hash~\cite{McEvoyTunstall07,BertoniDaemenAssche09}.
    \end{enumerate}
\item \textbf{Ataques e contramedidas para ECC (20 pgs).}
    \begin{enumerate}
        \item N\'iveis em que os ataques SCA podem ser aplicados;
        \begin{enumerate}
            \item Opera\c{c}\~oes na curva;
            \item Opera\c{c}\~oes no protocolo criptogr\'afico;
            \item Transfer\^encia da chave entre diferentes mem\'orias;
            \item Protocolo em n\'ivel de aplica\c{c}\~ao;
        \end{enumerate}
        \item Ataques do tipo SPA e contramedidas~\cite{danger2013synthesis};
        \begin{enumerate}
            \item Ataque SPA clássico;
            \item Ataques baseados em templates~\cite{medwed2008template};
            \item Ataques horizontais baseados em cross-correlation~\cite{witteman2011defeating};
            \item Ataques horizontais não-supervisionados baseados em clustering~\cite{heyszl2013clustering, perin2015semi};
            \item Aplica\c{c}\~ao de contramedidas em algoritmo esquerda para direita inseguro;
            \item Implementa\c{c}\~oes de tempo constante;
            \item Implementa\c{c}\~oes resistentes ao SPA;
            \item Impacto das contramedidas no desempenho~\cite{coron1999resistance,danger2013synthesis};
        \end{enumerate}
        \item Efic\'acia de implementa\c{c}\~ao de tempo constante;
        \begin{enumerate}
            \item Outros métodos para inviabilizar ataques por tempo;
        \end{enumerate}
        \item Recuperação de chaves com erros; ~\cite{VeyratCharvillon2013,LangeVrendendaalWakker2014}
        \begin{enumerate}
            \item Forma de pontua\c{c}\~ao de confiança de um SCA;
            \item Pontua\c{c}\~ao em implementa\c{c}\~oes protegidas;
            \item Solu\c{c}\~ao para pontua\c{c}\~oes baixas de confiabilidade.
        \end{enumerate}
    \end{enumerate}
\item \textbf{Ferramentas (10 pgs).}
    \begin{enumerate}
        \item Ferramentas para verifica\c{c}\~ao de c\'odigo de alto n\'ivel em  relação \`a execu\c{c}\~ao em tempo constante~\cite{Langley2012};
        \item Ferramentas para an\'alise de c\'odigo assembly em relação à  prote\c{c}\~ao contra ataques de pot\^encia;
            \begin{enumerate}
                \item Aplica\c{c}\~ao semi-autom\'atica de contramedidas~\cite{MossOswald2012};
            \end{enumerate}
        \item M\'etodos emp\'iricos para an\'alise de vazamentos;
            \begin{enumerate}
                \item Test Vector Leakage Assessment (TVLA) e suas limita\c{c}\~oes~\cite{GoodwillJun2011, WittemanJaffe2013, TunstallGoodwill2016};
            \end{enumerate}
        \item Ferramentas para recuperação de chaves com erros~\cite{VeyratCharvillon2013,LangeVrendendaalWakker2014}.
    \end{enumerate}
\end{enumerate}

\section{Detalhamento da Estrutura}
\begin{enumerate}
\item \textbf{Introdução.} Introdu\c{c}\~ao à \'area de seguran\c{c}a, criptografia sim\'etrica e assim\'etrica, e a seguran\c{c}a na Internet das Coisas (IoT)

\item \textbf{Encripta\c{c}\~ao sim\'etrica e hash.} Apresenta\c{c}\~ao da encripta\c{c}\~ao sim\'etrica e hash, mas com um breve tratamento sobre o assunto. Nesse contexto ser\~ao apresentados um algoritmo sim\'etrico (AES) e duas funções de hash (SHA-2, SHA-3).

\item \textbf{Criptografia de curvas el\'ipticas (ECC).} Introdu\c{c}\~ao \`a matem\'atica utilizada em criptografia de curvas el\'ipticas. Ser\~ao apresentadas curvas padronizadas pelo NIST. Al\'em disso, diferentes formas de implementar curvas el\'ipticas, visando melhor desempenho, serão discutidas. E, por fim, protocolos de curvas el\'ipticas utilizados em IoT.

\item \textbf{Ataques por canais laterais.} Ser\~ao apresentados alguns ataques de canais laterais, como de tempo e de pot\^encia. No caso do de tempo \'e necess\'ario avaliar o impacto na utiliza\c{c}\~ao em aplica\c{c}\~oes de IoT; al\'em disso, expor as limita\c{c}\~oes desses ataques. J\'a os ataques de pot\^encia podem diferir  dependendo de como analisam o perfil de energia, podendo ser simples (SPA), diferenciais (DPA), ou de ordem mais alta (High-order DPA).

\item \textbf{Ataques e contramedidas para encripta\c{c}\~ao sim\'etrica e hash.} Nesse item iremos explorar a utiliza\c{c}\~ao de ataques de canal lateral em algoritmos que n\~ao s\~ao de ECC, como por exemplo AES. Isso mostra que um mesmo ataque pode ser utilizado em algoritmos diferentes, e at\'e mesmo baseados em problemas matem\'aticos diferentes. Al\'em disso, ser\'a apresentada uma contramedida ao ataque explorado.

\item \textbf{Ataques e contramedidas para ECC.} Ser\~ao apresentados ataques e suas contramedidas em algoritmos ECC. Isso ser\'a feito considerando o problema matem\'atico utilizado em curvas el\'ipticas. Para que seja poss\'ivel ter uma vis\~ao mais abrangente das contramedidas \'e necess\'ario avaliar o custo de mem\'oria e desempenho da aplica\c{c}\~ao das mesmas. Dependendo do sistema que utiliza algoritmos criptogr\'aficos, como sensores, pode n\~ao ser poss\'ivel fazer o uso de determinada contramedida. E, por fim, ser\'a apresentada a forma de pontua\c{c}\~ao dos bits da chave obtidos pela an\'alise de tempo.

\item \textbf{Ferramentas.} Para facilitar a implementa\c{c}\~ao de um c\'odigo seguro \'e poss\'ivel utilizar ferramentas que analisam o algoritmo e o tornam seguro, ou apenas apresentam as vulnerabilidades. Tamb\'em ser\~ao apresentados os m\'etodos para an\'alise do algoritmo que essas ferramentas podem utilizar.

\end{enumerate}

\bibliographystyle{sbc}
\bibliography{sbc-template}

\pagebreak 

\section{Curricula Vitae}
\subsection{Lucas Zanco Ladeira}
\subsubsection{Profissional}
2013 - 2014: Boa Vista Servi\c{c}os (Estagi\'ario).\\
Trabalhei com desenvolvimento web na linguagem Java (JEE). Foram utilizados frameworks como Spring, Struts 2, e Hibernate. Outras bibliotecas para facilitar o desenvolvimento foram utilizadas como JSTL. No projeto foi utilizado o motor de regras guvnor drools e servidores de aplica\c{c}\~ao JBoss e Tomcat.

\subsubsection{Educa\c{c}\~ao}
\begin{itemize}
\item 2014 - 2015: Bolsa de inicia\c{c}\~ao cient\'ifica FAPESP\\
A partir do meio do ano de 2014 fui contemplado com uma bolsa de inicia\c{c}\~ao cient\'ifica da FAPESP, para desenvolvimento de uma aplica\c{c}\~ao offline e segura de meio de pagamento com java card.\\
Durante esse trabalho foram pesquisados algoritmos criptogr\'aficos diferentes (AES, RSA, Esquema de Coron de Criptografia Homom\'orfica, Hash Sha-2) e t\'ecnicas de cria\c{c}\~ao de sess\~ao segura. Al\'em disso, formas diferentes de autentica\c{c}\~ao como PIN, biometria digital, e \textit{challenge-response authentication}.

\item 2016: Bolsa de mestrado da FAPESP\\
O projeto de mestrado tem o intuito de estudar algoritmos criptogr\'aficos, e suas implementa\c{c}\~oes seguras contra ataques de canal lateral. Isso torna necess\'ario o estudo dos diferentes ataques, e as contramedidas eficientes sobre cada uma. Al\'em disso, cada contramedida apresenta um n\'ivel de dificuldade diferente para implementa\c{c}\~ao, custo de processamento e mem\'oria.

\end{itemize}

\subsubsection{Publica\c{c}\~oes}
Durante o desenvolvimento da inicia\c{c}\~ao cient\'ifica as seguintes publica\c{c}\~oes foram conquistadas:\\
\begin{itemize}
\item Mecanismos de Autenticação em Smart Card utilizando Criptografia Totalmente Homomórfica. XXII Iberchip Workshop, 2016.
\item Autenticação em Java Card com Criptografia Homomórfica, V Congresso de Pesquisa Científica: Inovação, Sustentabilidade, Ética e Cidadania, 2015.
\item Meio de Pagamento Seguro e Off-line Utilizando Tecnologia Java Card. VII Congresso de Iniciação em Desenvolvimento Tecnológico e Inovação da UFSCar, 2014.
\end{itemize}

\subsection{Erick Nogueira do Nascimento}

\subsubsection*{Formação}
\begin{itemize}\setlength\itemsep{1pt}
\item 2004 a 2008, Bacharelado em Engenharia de Computação, UNICAMP.
\item 2009 a 2011, Mestrado em Ciência da Computação, UNICAMP.
\item 2011 (cursando), Doutorado em Ciência da Computação, UNICAMP.
\end{itemize}

\subsubsection*{Histórico Profissional}

É candidato ao doutorado em Ciência da Computação na UNICAMP. Atualmente realiza um estágio de pesquisa na empresa Riscure BV nos Países Baixos, com o tema de ataques por canais laterais de potência e radiação eletromagnética contra implementações em software de algoritmos criptográficos assimétricos. Foi bolsista do Ciência sem Fronteiras por um ano, de Abril de 2015 a Abril de 2016, modalidade de estágio de doutorado sanduíche no exterior, na Radboud University Nijmegen, Países Baixos, sob orientação do Prof. Peter Schwabe. Foi pesquisador no CPqD de 2012 a 2014, onde atuou principalmente em projetos envolvendo criptografia em software e hardware, bem como engenharia reversa de software, análise de vulnerabilidades e testes de intrusão em aplicações. 

\subsubsection*{Publicações Recentes}

\begin{enumerate}\setlength\itemsep{1pt}
    \item Nascimento, E., López, J. and Dahab, R., "Efficient and secure elliptic curve cryptography for 8-bit avr microcontrollers." \textsl{International Conference on Security, Privacy, and Applied Cryptography Engineering (SPACE)}. Springer International Publishing, 2015. 
    \item Kawakami, H., Gallo, R., Dahab, R., \& Nascimento, E., “Hardware Security Evaluation Using Assurance Case Models”. In \textsl{Availability, Reliability and Security (ARES), 2015 10th International Conference on} (pp. 193-198). IEEE, 2015.
    \item Nascimento, E., Abarzua, R., Lopez, J., Dahab, R., “A comparison of simple side-channel analysis countermeasures for variable-base elliptic curve scalar multiplication”. \textsl{Proceedings of the Brazilian Symposium on Information and Systems Security}, 2014.
\end{enumerate}

\subsection{João Paulo Fernandes Ventura}

\subsubsection{Formação}
\begin{itemize}\setlength\itemsep{1pt}
\item 2003 a 2007, Graduação em Engenharia de Computação, UNICAMP.
\end{itemize}

\subsubsection{Histórico Profissional}
\begin{itemize}\setlength\itemsep{1pt}
\item 2010 à 2012, \textit{Software Engineer} na IBM, atuando no desenvolvimento de virtual appliances para \textit{RHEV Blue}.
\item 2013 à 2013, \textit{Software Engineer} na Intel Corporation, atuando no desenvolvimento do Tizen OS.
\item 2013 à 2014, \textit{Software Developer} no Samsung Instituto de Desenvolvimento para a Informática, atuando no desenvolvimento do Samsung Knox.
\end{itemize}

\subsubsection{Publicações Recentes}
\begin{itemize}\setlength\itemsep{1pt}
\item Ventura, J.P.F., Dahab, R., "Introduction to Side-Channel Attacks". In \textit{IX Brazilian Symposium on Information and Computational Systems Security}, 2009.
\end{itemize}

\subsection{Ricardo Dahab}
\subsubsection{Formação}
\begin{itemize}
\item 1978 - Bacharelado em Ciência da Computação, UNICAMP
\item 1984 - Mestrado em Ciência da Computação, UNICAMP
\item 1993 - Ph.D. em Combinatória e Otimização, University of Waterloo, Canadá
\item 2002 - Livre-docência em Complexidade de Algoritmos, UNICAMP
\end{itemize}
\subsubsection{Histórico Profissional}
Professor em tempo integral na UNICAMP desde 1982, com promoções em 1984 para professor assistente, depois em 1993 para assistente-doutor, e em 2002 para professor associado. 

\subsubsection{Publicações Recentes}
Link para CV Lattes: \texttt{http://lattes.cnpq.br/9093331241572944}
\begin{enumerate}\setlength\itemsep{1pt}
    \item Nascimento, E., López, J. and Dahab, R., "Efficient and secure elliptic curve cryptography for 8-bit avr microcontrollers." \textsl{International Conference on Security, Privacy, and Applied Cryptography Engineering (SPACE)}. Springer International Publishing, 2015. 
    \item Kawakami, H., Gallo, R., Dahab, R., \& Nascimento, E., “Hardware Security Evaluation Using Assurance Case Models”. In \textsl{Availability, Reliability and Security (ARES), 2015 10th International Conference on} (pp. 193-198). IEEE, 2015.
    \item Nascimento, E., Abarzua, R., Lopez, J., Dahab, R., “A comparison of simple side-channel analysis countermeasures for variable-base elliptic curve scalar multiplication”. \textsl{Proceedings of the Brazilian Symposium on Information and Systems Security}, 2014.
\end{enumerate}

\subsection{Diego F. Aranha}

\subsubsection*{Formação}
\begin{itemize}\setlength\itemsep{1pt}
\item 2000 a 2005, Graduação em Ciência da Computação, Universidade de Brasília.
\item 2005 a 2007, Mestrado Ciência da Computação, UNICAMP.
\item 2007 a 2011, Doutorado em Ciência da Computação pela UNICAMP.
\end{itemize}

\subsubsection*{Histórico Profissional}

É Professor Doutor na Universidade Estadual de Campinas (Unicamp) desde 2014. Tem experiência na área de Criptografia e Segurança Computacional, com ênfase em implementação eficiente de algoritmos criptográficos e análise de segurança de sistemas reais. Coordenou a primeira equipe de investigadores independentes capaz de detectar e explorar vulnerabilidades no software da urna eletrônica em testes controlados organizados pelo Tribunal Superior Eleitoral. É Bacharel em Ciência da Computação pela Universidade de Brasília (2005), Mestre (2007) e Doutor (2011) em Ciência da Computação pela Universidade Estadual de Campinas. Foi doutorando visitante por 1 ano na Universidade de Waterloo, Canadá, e Professor Adjunto por pouco mais de 2 anos no Departamento de Ciência da Computação da Universidade de Brasília. É membro do Comitê Consultivo da Conferência Internacional em Criptografia e Segurança da Informação na América Latina (LATINCRYPT) e da Comissão Especial de Segurança da Sociedade Brasileira de Computação (CESEG), responsável pelo Simpósio Brasileiro de Segurança da Informação e Sistemas Computacionais (SBSEG), tendo coordenado o Comitê de Programa na edição 2014 de ambos os eventos. Recebeu em 2015 os prêmios Google Latin America Research Award para pesquisa em privacidade e Inovadores com Menos de 35 Anos Brasil da MIT Technology Review por seu trabalho com o voto eletrônico.

\subsubsection*{Publicações relevantes}

\begin{enumerate}\setlength\itemsep{1pt}
    \item T. Oliveira, J. López, {D. F. Aranha}, F. Rodr\'iguez-Henr\'iquez, ``Lambda coordinates for binary elliptic curves'',
    In \textsl{15th International Workshop on Cryptographic Hardware and Embedded Systems (CHES 2013)}, Springer LNCS 8086, pp. 311--330, Santa Barbara, USA, 2013. \textbf{Best Paper Award!}

    \item L. B. Oliveira, {D. F. Aranha}, C. P. L. Gouvêa, M. Scott, D. F. Câmara, J. López, R. Dahab.
    ``TinyPBC: Pairings for Authenticated Identity-Based Non-Interactive Key Distribution in Sensor Networks'', 
    \textsl{Computer Communications}, Vol. 34, Issue 3, pp. 485--493, 2011.

    \item {D. F. Aranha}, K. Karabina, P. Longa, C. H. Gebotys, J. López.
    ``Faster Explicit Formulas for Computing Pairings over Ordinary Curves'',
    In \textsl{30th International Conference on the Theory and Applications of Cryptographic Techniques (EUROCRYPT 2011)}, Springer LNCS 6632, pp. 48--68, Tallinn, Estonia, 2011.
\end{enumerate}

\subsection{Julio L\'opez}

\subsubsection*{Formação}
\begin{itemize}\setlength\itemsep{1pt}
\item 1979 a 1983, Graduação em Matemática, Universidad del Valle.
\item 1983 a 1985, Mestrado  em Matemática Aplicada, Universidad del Valle.
\item 1995 a 2000, Doutorado em Ciência da Computação no Instituto da Computação da UNICAMP com período sanduíche em University of Waterloo.
\item 2009, Livre-docência pela UNICAMP.
\end{itemize}

\subsubsection*{Histórico profissional}

Professor Associado no Instituto de Computação da Universidade Estadual de Campinas. Tem experiência na área de Ciência da Computação, com ênfase em Engenharia Criptográfica, atuando principalmente nos seguintes temas: algoritmos criptográficos, implementação hardware/software de criptossistemas de curvas elípticas, bibliotecas criptográficas, aritmética computacional e aplicações criptográficas.

\subsubsection*{Publicações relevantes}
\begin{enumerate}\setlength\itemsep{1pt}
\item Danilo Câmara, Conrado PL Gouvêa, Julio López, and Ricardo Dahab. \emph{Fast software polynomial multiplication on arm processors using the neon engine}. In Security Engineering and Intelligence Informatics, pages 137-154. Springer, 2013.

\item D. F. Aranha, J. López, and D. Hankerson. \emph{Efficient Software Implementation of Binary Field Arithmetic Using Vector Instruction Sets}. In M. Abdalla and P. S. L. M. Barreto, editors, The First International Conference on Cryptology and Information Security (LATINCRYPT 2010), volume 6212 of LNCS, pages 144-161, 2010.

\item Conrado PL Gouvêa, Leonardo B Oliveira, and Julio López. \emph{Efficient software implementation of public-key cryptography on sensor networks using the msp430x microcontroller}. Journal of Cryptographic Engineering, 2(1):19-29, 2012.
\item C. P. L. Gouv\^ea and J. L\'opez. Implementing GCM on armv8. In Topics in Cryptology - CT-RSA 2015, The Cryptographer's Track at the RSA Conference 2015, San Francisco, CA, USA, April 20-24, 2015. Proceedings, pages 167,180, 2015.
\end{enumerate}

\end{comment}

\end{document}
