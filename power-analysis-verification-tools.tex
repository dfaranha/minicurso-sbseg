\subsection{Ferramentas para verificação de implementações contra ataques de potência}

%\subsubsection{Aplicação semi-automática de contramedidas}

%==================================================================================================
%% texto proposta projeto Intel-Fapesp "Secure execution of cryptographic algorithms" de 2014
%==================================================================================================

%Formal verification also has been applied to power analysis. Moss et al.~\cite{moss2011automatic} introduce an algorithm to  automate the process of application of masking countermeasures against DPA, at the assembly code level. 

%%% Formal verification of software implementations of cryptographic algorithms against power analysis has been studied.
Verificação formal de implementações em software de algoritmos criptográficos contra análise de potência é um assunto que tem sido pesquisado recentemente.
%
% Maggi et al.~\cite{maggi2013automated, Agosta:2013:CSC:2463209.2488833} propose a method based on data flow analysis, to be able to identify dependencies of each instruction on secret data. 
%
Por exemplo, Maggi et al.~\cite{maggi2013automated, Agosta2013} propuseram um método baseado em análise de fluxo de dados para identificar dependências entre as instruções executadas e dados secretos.
%
% The method is implemented into the LLVM compiler as a specialized pass that works at the intermediate representation level, thus it is architecture agnostic, supporting any of the architectures supported by LLVM. The tool automatically instantiates the essential masking countermeasures. 
%
O método é implementado em um compilador LLVM como uma passada especializada operando no nível de representação intermediária, portanto ela é agnóstica em arquitetura, suportando quaisquer das arquiteturas de computador suportadas por LLVM. A ferramenta automaticamente instancia contramedidas de masking à implementação.
%

% Most recently, Bayrak et al~\cite{BayrakRegazzoniNovo:2013, Bayrak2014} showed how to reduce the verification problem, in the case of power channels, into a set of Boolean satisfiability problems, which can be efficiently handled by current SAT solvers, and which overcome some of shortcomings of the information flow analysis approach. 

Mais recentemente, Bayrak et al~\cite{BayrakRegazzoniNovo2013, Bayrak2014} mostraram como reduzir o problema de verificação da resistência de uma implementação a vazamento por canais de potência à um conjunto de problemas SAT, os quais podem ser eficientemente tratados pelos resolvedores SAT atuais. Tal método a princípio endereça as limitações da abordagem baseada em análise de fluxo de informação.

% Their approach motivated the introduction of satisfiability module theories (SMT) solvers by Eldib et al~\cite{EldibWang2014, EldibWang2014_QMS, EldibWang2014_SMT, EldibWang2014_sc_sniffer} to tackle the problem in the case of power analysis. The latter authors also proposed methods for the automated application of countermeasures.
%
Resolvedores baseados em teorias do módulo de satisfabilidade (SMT) foram aplicados por Eldib et al~\cite{EldibWang2014, EldibWang2014_QMS, EldibWang2014_SMT, EldibWang2014_sc_sniffer} a este problema. Estes últimos também propuseram métodos para a aplicação automatizada de contramedidas.
