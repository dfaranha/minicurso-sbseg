\section{Introdução}

A Internet das Coisas permite a interconexão de múltiplos dispositivos embarcados que manipulam dados de diferentes tipos e realizam tarefas diversas, algumas até críticas. Este ambiente promete trazer enormes benefícios para a vida em sociedade e, naturalmente, induz novos requisitos para sua implementação eficaz e robusta. Entre estes requisitos, segurança, tolerância a falhas e privacidade surgem como dimensões novas e fundamentais no projeto de sistemas embarcados.

Apesar da importância de se equipar a Internet das Coisas com mecanismos robustos de segurança, a mudança de paradigma trazida por essa nova tecnologia cria um problema desafiador para o projeto de mecanismos de segurança: enquanto os dispositivos precisam permanecer compactos e baratos, a quantidade massiva de dados coletados e transportados por esses dispositivos e sua natureza sensível certamente terão implicações significativas em privacidade. Simultaneamente, qualquer solução prática precisa levar em conta os recursos reduzidos e a proteção física limitada que são típicas de dispositivos da Internet das Coisas.

Este minicurso discute técnicas para implementação segura de algoritmos criptográficos, com aplicações para a proteção de dispositivos embarcados operando na Internet das Coisas. Pretende-se discutir duas classes de algoritmos, criptografia simétrica baseada em cifras de bloco e funções de resumo criptográfico e criptografia assimétrica baseada em curvas elípticas, abrangendo seu projeto e implementação eficiente e segura contra ataques de canal lateral. A motivação para o estudo de primitivas simétricas é o menor consumo de recursos que seus correspondentes assimétricos, resultando em maior eficiência e tempo de vida quando sua utilização é maximizada em protocolos de comunicação para a Internet das Coisas. Por outro lado, esquemas para criptografia assimétrica são essenciais para estabelecer chaves criptográficas para primitivas simétricas e, portanto, precisam ser parte integral para qualquer arquitetura aplicada de segurança. Nesse cenário, o estudo de esquemas assimétricos baseados em curvas elípticas são de interesse especial, devido ao seu potencial para implementação em dispositivos com recursos restritos.


%% Seções da introdução (rascunho)

%%%%%% Segurança da informação; criptografia; segurança em IOT

% Criptografia Simetrica / Hash / Assimetrica
% Criptografia de curvas elípticas
% Permite SCA / diferentes ataques abordados
% Ataques e contramedidas Simetrico / ECC
% Segurança em IoT