\subsection{Níveis em que os ataques SCA podem ser aplicados}

\begin{comment}
\subsubsection{Operações na curva}
\subsubsection{Operações no protocolo criptográfico}
\subsubsection{Transferência da chave entre diferentes memórias}
\subsubsection{Protocolo em nível de aplicação}
\end{comment}

%TODO 1 parag. p/ explicar a figura

%TODO: Figura com piramide de ataques SCA.

%TODO: "Transferência da chave entre diferentes memórias"
%TODO: Encontrar uma ref. sobre um SCA bem sucedido deste tipo

%TODO: Protocolo em nivel de aplicacao

\subsection{Ataques do tipo SPA e contramedidas}

\subsubsection{Ataque SPA clássico}

%TODO
\begin{comment}
1.	Ataque SPA à alg. ECSM binário left-to-right (Dbl-and-Add not Always)
	a.	Se impl não é de tempo constante, então é possível realizar ataque de tempo.
		i.	P.ex., se usa if and else, então pode-se determinar a cada iteração qual bloco, if ou else, é tomado.
b.	Se é de tempo constante, SPA (com ou sem power model) pode ser aplicado para distinguir os padrões no trace das iterações com apenas DBL (bit=0) daquelas com DBL+ADD (bit=1).
\end{comment}

%TODO 2. Argumentar que é fortemente desejável que as implementações sejam de tempo constante, de modo a ser uma base para a implementação das outra contramedidas para power analysis. 

\begin{comment}
	3.	Ataque SPA à alg. ECSM Dbl-and-Add Always
	a.	Let’s say that R_0 is the dummy register.
	b.	Then, if i==0, then a dummy addition is performed, with the output stored in R_i
	i.	R_i  = R_i + P 
	ii.	OR   R_i = R_{1-i} + P
	c.	%%% TODO %%%: explicar como atacar com SPA.
\end{comment}

\subsubsection{Contramedidas}


\begin{comment} % === CONTRAMEDIDAS ===
4.	TODO: apresentar uma contramedida de cada vez entre subseções com ataques, mostrando como ela protege contra um determinado ataque.
	a.	CM1 - Scalar Randomization (SR)
	b.	CM2 - Proj. Coord. Randomization and Re-randomization (CRR)
	c.	CM3 - Point Blinding (PB) 
	d.	CM4 – Scalar Splitting (SS)
\end{comment}


\begin{comment} % === Ataques SCA contra ECC devem ser do tipo single-trace  ===
5.	Explicar porque no contexto de PKC (RSA e ECC) não fazem sentido ataques que envolvem mais de um trace, como p.ex., DPA.
\end{comment}


\begin{comment}  % === ESTRUTURA ORIGINAL ===
\subsubsection{Ataques baseados em templates}
\subsubsection{Ataques horizontais baseados em cross-correlation}
\subsubsection{Ataques horizontais não-supervisionados baseados em clustering}

\subsubsection{Aplicação de contramedidas em algoritmo esquerda para direita inseguro}
\subsubsection{Implementações de tempo constante}
\subsubsection{Implementações resistentes ao SPA}
\subsubsection{Impacto das contramedidas no desempenho}

\subsection{Eficácia de implementação de tempo constante}
\subsubsection{Outros métodos para inviabilizar ataques por tempo}
\end{comment}


\subsection{Ataque SPA à alg. ECSM atômico}
%	a.	TODO: quais formulas usar?

\subsection{Ataque SPA a alg. Mont Ladder com SR}

\subsection{Ataque template SPA à alg. Mont Ladder c/ SR+CRR}
%TODO: Copy-and-paste do meu paper no SAC 2016.

\subsection{Ataque HCA à alg. Mont Ladder c/ SR+CRR}
%TODO: Copy-and-paste meu texto sobre HCA.

\subsection{Ataque template versus Ataque horizontal}
%TODO: Ataques template versus HCA: vantagens e desvantagens de cada um.
% a.	Aplicabilidade: até então estes ataques só foram demonstrados em MCUs embarcados de 8, 16 e 32 bits, devido ao alto nível de SNR nos traces de potência e EM capturados nestas CPUs. 
