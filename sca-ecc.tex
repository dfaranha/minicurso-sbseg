\subsection{Níveis em que os ataques SCA podem ser aplicados}

\begin{comment}
\subsubsection{Operações na curva}
\subsubsection{Operações no protocolo criptográfico}
\subsubsection{Transferência da chave entre diferentes memórias}
\subsubsection{Protocolo em nível de aplicação}
\end{comment}

%TODO 1 parag. p/ explicar a figura

%TODO: Figura com piramide de ataques SCA.

%TODO: "Transferência da chave entre diferentes memórias"
%TODO: Encontrar uma ref. sobre um SCA bem sucedido deste tipo

%TODO: Protocolo em nivel de aplicacao

\subsection{Ataques de tempo e SPA ao algoritmo double-and-add-not-always}

\subsubsection{Ataque de tempo ao algoritmo double-and-add-not-always}
Considerando que o SPA de tempo se baseia na variação de tempo do algoritmo dependendo do valor da chave é possível realizar o ataque de tempo em algoritmos que não executam em tempo constante. Nesse caso, a maneira mas simples de observar uma implementação insegura é a presença de branchs condicionais. Um exemplo é no caso de um algoritmo possuir uma instrução de \textit{if} onde ele apenas executa as operações contidas nessa instrução dependendo dos \textit{bits} da chave privada.

Nesse contexto podemos observar um ataque de tempo aplicado a uma implementação do RSA por \cite{Kocher96}, esse foi o primeiro ataque de tempo proposto. O mesmo pode ser aplicado a implementações de curvas elípticas considerando que basicamente o atacante faz a análise do tempo de execução do algoritmo, tenta advinhar os \textit{bits} da chave e valida se o resultado está correto.

No trabalho de \cite{danger2013synthesis} é descrito um ataque à um implementação genérica de ECSM. O mesmo segue os seguintes passos: primeiramente o atacante coleta o tempo de execução de diferentes ECSMs com o mesmo escalar e diferentes pontos base. Para cada ECSM, ele simula a computação usando um simulador de software com exatamente a mesma implementação do chip alvo, "chutando" o valor do bit $i$ do escalar. 

Agora suponha, sem perda de generalidade, que a hipótese é de que o valor do \textit{bit} é $0$.  Ele separa os diferentes tempos de execução em dois conjuntos, $S_1$ e $S_2$. Caso uma redução é necessária ao final da execução o tempo obtido é armazenado em $S_2$, senão é armazenado em $S_1$. Após toda a execução é feita uma média dos tempos armazenados em $S_1$ e $S_2$, e se a diferença entre as médias for aproximadamente o tempo de execução da redução portanto a hipótese estava correta.

\begin{comment}
\erick[inline]{Lucas: ver o comentario no fonte. Converter comentario em texto}.

1.	Ataque SPA à alg. ECSM binário left-to-right (Dbl-and-Add not Always)
	a.	Se impl não é de tempo constante, então é possível realizar ataque de tempo.
		i.	P.ex., se usa if and else, então pode-se determinar a cada iteração qual bloco, if ou else, é tomado.
	b. O ataque de tempo em~\cite{Kocher96} ao RSA pode ser aplicado no contexto de ECC. Segue abaixo a ideia do ataque (baseada no survey de ~\cite{Danger2013}, Sec. 3.2.1).
	
	O atacante coleta o tempo de execução de diferentes ECSMs com o mesmo escalar e diferentes pontos base. Para cada ECSM, ele simula a computação usando um simulador de software com exatamente a mesma implementação do chip alvo, "chutando" o valor do bit $i$ do escalar. Suponha, sem perda de generalidade, que a hipótese é de que o valor do bit é 0.  Ele separa os diferentes tempos de execução em dois conjuntos, S_1 e S_2. Se, a iteração
	
	%TODO: CONT HERE
	

[Kocher96] Kocher, P.C.: Timing attacks on implementations of Diffie–
Hellman, RSA, DSS, and other systems. In: Proceedings of CRYPTO’96, LNCS, vol. 1109. Springer, Berlin, pp. 104–113
(1996)
\end{comment}

\subsubsection{Ataque SPA ao algoritmo double-and-add-not-always}
O algoritmo \textit{double-and-add-not-always} executa em tempo constante, contudo o ataque SPA (com ou sem power model) pode ser aplicado para distinguir os padrões no \textit{trace} das iterações com apenas DBL (onde o \textit{bit} da chave é igual à $0$) daquelas com DBL+ADD (com \textit{bit} da chave igual à $1$). Um ataque deste tipo segmenta/divide o \textit{trace} de potência de uma execução do ECSM em \textit{subtraces}, cada uma contendo uma operação de ponto (ADD ou DBL). 

Se o tempo de execução de uma operação de ADD for diferente do tempo de DBL, então o comprimento dos \textit{subtraces} revela onde estão os ADDs e consequentemente os bits do escalar. Se o tempo da operação ADD e o tempo da DBL forem iguais, e uma fórmula unificada para ADD and DBL é utilizada, então, se for aplicada correlação (coeficiente de correlação de Pearson) entre todos os pares de \textit{subtraces}, o resultado será que a correlação será mais alta para os pares de subtraces cuja operação correspondente é a mesma (i.e., (ADD,ADD) ou (DBL,DBL)), identificando portanto as operações de ponto e consequentemente os \textit{bits} do escalar.

Mesmo sendo que a implementação seja vulnerável a SPA é desejável que execute em tempo constante, o que cria uma base para a implementação de outras contramedidas. Portanto, é necessário observar que para executar o ataque SPA são requisitados equipamentos para medir a potência como um osciloscópio, além disso, o ataque por tempo possui simplicidade em comparação com SPA sendo que apenas é necessária a capacidade de verificar o tempo de execução do algoritmo e inserir entradas distintas.

\begin{comment}
\erick[inline]{Lucas: ver o comentario no fonte. Converter comentario em texto}.

b.	Se é de tempo constante, SPA (com ou sem power model) pode ser aplicado para distinguir os padrões no trace das iterações com apenas DBL (bit=0) daquelas com DBL+ADD (bit=1). Um ataque deste tipo segmenta/divide o trace de potencia de uma execucao do ECSM em subtraces, cada um contendo uma operacao de ponto (ADD ou DBL). Se as tempo(ADD) != tempo(DBL), então o comprimento dos subtraces revela onde estão os ADDs e consequentemente os bits do escalar. Se tempo(ADD) == tempo(DBL) e uma formula unificada para ADD and DBL é utilizada, então, se for aplicada correlação (coeficiente de correlacao de Pearson) entre todos os pares de subtraces, o resultado será que a correlação será mais alta para os pares de subtraces cuja operação correspondente é a mesma (i.e., (ADD,ADD) ou (DBL,DBL)), identificando portanto as operações de ponto e consequentemente os bits do escalar.
\end{comment}

%TODO 2. Argumentar que é fortemente desejável que as implementações sejam de tempo constante, de modo a ser uma base para a implementação das outra contramedidas para power analysis.

\subsection{Double-and-add-always algorithm of Coron~\cite{Coron1999}}

\begin{comment}
\erick[inline]{Lucas: traduzir esta secao}
\end{comment}

O algoritmo {\it double-and-add-always} de \cite{Coron1999} (Algoritmo 8) utiliza uma adição de ponto falsa conhecida como \textit{dummy point addition} quando o \textit{bit} escalar $k_i$ é $0$, tornando a sequência de operações computadas durante a multiplicação escalar independente do valor do escalar.

\begin{algorithm}[h] %\scriptsize %\footnotesize
	\caption{\small{\textit{Double-and-add always} algorithm resistant against SPA}}
	\label{Double-and-add-Coron}
	\begin{algorithmic}[1]
		\REQUIRE  Point $\textbf{P} \in E(\mathbb{F}_q),$ $k=(k_{n-1},\ldots,k_1,k_0)_2 \in \mathbb{N}$
		\ENSURE  $Q=[k] \cdot P$\\
		\STATE $R_0\leftarrow P_{\infty}$   \\
		\FOR{$i$ \textbf{from} $n-1$ \textbf{to} $0$} 
		\STATE $R_0\leftarrow 2R_0$  \\
		\STATE $R_1\leftarrow R_0+P$\\ \label{Paso_R_1_Double-and-add-Coron}
		\STATE $R_0\leftarrow R_{k_i}$\label{Step5Double-and-add-Coron} \\
		\ENDFOR
		\STATE return $R_0$\\
	\end{algorithmic}
\end{algorithm}

Portanto o adversário não consegue, em princípio, advinhar o valor do \textit{bit} $k_i$ por SPA. Uma desvantagem desse método é a sua baixa eficiência. Ele requer $nA + nD$ operações no corpo, um aumento de $33\%$ nas operações do corpo em comparação a versão desprotegida binária do algoritmo \textit{left-to-right}.

\subsubsection{Fouque and Valette's Doubling Attack \cite{CHES:FouVal03}}\label{Fouque-Valette-DoublingAttack}
O ataque de duplicação de Fouque-Valette~\cite{CHES:FouVal03} é baseado no fato de que é possível detectar se dois valores intermediários são iguais quando o algoritmo calcula a multiplicação escalar para pontos escolhidos $P$ e $2P$. Diversos algoritmos protegidos contra SPA são vulneráveis ao ataque d eFouque e Valette's, como o algoritmo clássico \textit{left-to-right} binário, incluindo as variações do mesmo, como o \textit{double-and-add-always} de Coron.

No algoritmo de Coron \textit{double-and-add-always}, algoritmo (Algoritmo~\ref{Double-and-add-Coron}), a soma parcial é calculada da seguinte maneira: $S_m(P) = \sum_{i=1}^{m}k_{n-i}2^{m-i} P=\sum_{i=1}^{m-1}k_{n-i}2^{m-1-i}(2P)+k_{n-m}P= S_{m-1}(2P)+k_{n-m}P$. Então o resultado intermediário do algoritmo no passo $m$ quando a entrada for $P$ será igual ao resultado intermediário no passo $m-1$ quando a entrada for $2P$, se e somente se, $k_{n-m} = 0$. Portanto, um atacante pode obter o escalar secreto comparando a computação de duplicação no passo $m+1$ para $P$ e no passo $m$ para $2P$ para recuperar o bit $k_{n-m}$. Se ambas as computações forem idênticas, $k_{n-m} = 0$, senão $k_{n-m} = 1$. Isso mostrou que com apenas duas requisições de multiplicação escalar escolhidas pelo atacante, é possível recuperar todos os \textit{bits} do escalar.~\footnote{O atacante coleta um traço de energia para a computação de $kP$ e um para a computação de $k(2P)$. Para cada iteração $m=1,...,n$ ele executa o ataque como descrito e encontra $k_{n-m}$.}

\begin{comment}
The doubling attack of Fouque-Valette~\cite{CHES:FouVal03} is based on the fact that it is possible to detect if two intermediate values are equal when the algorithm computes the scalar multiplication for points chosen points $P$ and $2P.$ Several algorithms protected against SPA are vulnerable to Fouque and Valette's attack, such as the classic binary left-to-right algorithm, including those derived from it, such as Coron's double-and-add-always algorithm.

%In ~\textit{double-and-add-always} algorithm (Algorithm~\ref{Double-and-add-Coron}), the partial sums are computed as follows: $S_k(P) = \sum_{i=0}^{k}d_{n-i}2^{k-i} P=\sum_{i=0}^{k-1}d_{n-i}2^{k-1-i}(2P)+d_{n-k}P= S_{k-1}(2P)+d_{n-k}P$. So, the intermediate result of the algorithm at step $k$ when given input $P$ will be equal to the intermediate result at step $k-1$ when given input $2P$, if and only if, $d_{n-k}=0$. Therefore, an attacker can obtain the secret scalar by comparing the doubling computation at step $k+1$ for $P$ and at step $k$ for $2P$ to recover the bit $d_{n-k}.$ If both computations are identical, $d_{n-k}=0$, otherwise $d_{n-k}=1$.
In Coron's double-and-add-always algorithm (Algorithm~\ref{Double-and-add-Coron}), the partial sums are computed as follows: $S_m(P) = \sum_{i=1}^{m}k_{n-i}2^{m-i} P=\sum_{i=1}^{m-1}k_{n-i}2^{m-1-i}(2P)+k_{n-m}P= S_{m-1}(2P)+k_{n-m}P$. So, the intermediate result of the algorithm at step $m$ when given input $P$ will be equal to the intermediate result at step $m-1$ when given input $2P$, if and only if, $k_{n-m}=0$. Therefore, an attacker can obtain the secret scalar by comparing the doubling computation at step $m+1$ for $P$ and at step $m$ for $2P$ to recover the bit $k_{n-m}.$ If both computations are identical, $k_{n-m}=0$, otherwise $k_{n-m}=1$. It has been shown that with only two scalar multiplication requests chosen by the attacker, it is possible to recover all the bits of the scalar.~\footnote{The attacker collects one power trace for the computation of $kP$ and one for the computation of $k(2P)$. For each iteration $m=1,...,n$, he runs the attack as described and finds $k_{n-m}$.}
%%\vspace{-0.5cm}
\end{comment}


\subsection{Contramedidas}

% TODO: apresentar uma contramedida de cada vez entre subseções com ataques, mostrando como ela protege contra um determinado ataque.

\begin{comment} % === CONTRAMEDIDAS ===
	a.	CM1 - Scalar Randomization (SR)
	b.	CM2 - Proj. Coord. Randomization and Re-randomization (CRR)
	c.	CM3 - Point Blinding (PB) 
	d.	CM4 – Scalar Splitting (SS)
\end{comment}

\subsubsection{Scalar Randomization (SR)}
	%	\begin{itemize}
	%\item 
	At the beginning of the scalar multiplication:
	\begin{enumerate}
		\item Randomly select $r\in_R \{0,1\}^n$, for a small $n$. $n=32$ seems to be a reasonable security/efficiency trade-off.
		\item Compute $k' \rcv k + r |E|$.
		\item Use $k'$ in place of $k$.
	\end{enumerate}
	
	%\item 
	\underline{Cost}: generate $n$ (pseudo-)random bits, $n$ iterations to the ECSM, MULs and ADDs to compute $k'$, $n$ bits of SRAM.\\
	%\item 
	
	\underline{Effectiveness}: vulnerable to Online Template Attacks (OTA) {[Batina,2014]}, among others.
	%	\end{itemize}

\subsubsection{Projective Coordinates (Re-)Randomization (CR)~\cite{Coron1999}}
As coordenadas projetivas usadas em Montgomery Ladder são $(X : Z)$. Durante o método ECSM primeiramente é necessário gerar um valor randômico $\lambda \mathbb{F}_p \backslash \{0\}$. Então calcule $Z_2 \rcv \lambda$ and $X_2 \rcv u \cdot \lambda$, onde $u$ é a coordenada $x$ do ponto $P$ da entrada. Então utilize $P'=(X_2:Z_2)$ no lugar de $P$.

Existe um custo de geração de $\ceil{log_2(p)}$ \textit{bits} aleatórios. Além disso, algumas vantagens ligadas a essa randomização das coordenadas como por exemplo a resistência à ataques online de template \cite{BatinaChmielewski2014}, pode ser aplicada para as curvas de Edwards, \textit{twisted} Edwards e Montgomery. Por fim, a cada iteração do método de ECSM ela pode ser aplicada, sendo apenas necessária a coordenada $x$ no caso de Montgomery Ladder.


\begin{comment}
\erick[inline]{Lucas: traduzir e escrever em texto corrido. Descrito para o caso de Montgomery Ladder somente com a coordenada $x$}

Original projective coordinates used in Mont Ladder: $(X:Z)$.\\
In the beginning of the ECSM, do:
\begin{enumerate}
	\item Generate random $\lambda \mathbb{F}_p \backslash \{0\}$.
	\item Do $Z_2 \rcv \lambda$ and $X_2 \rcv u \cdot \lambda$, where $u$ is the $x$-coordinate of input point $P$.
	\item Use $P'=(X_2:Z_2)$ in place of $P$.
\end{enumerate}	
\underline{Cost}: generation of $\ceil{log_2(p)}$ random bits, 1 M.\\
\underline{Effectiveness}: 
\begin{itemize}
	\item Resistant to Online Template Attacks~\cite{BatinaChmielewski2014}.
	\item Can also be applied to Edwards, twisted Edwards and Montgomery curves.
	\item Can also be used at every ECSM iteration (re-randomization).
\end{itemize}
\end{comment}

\subsubsection{Point Blinding (PB)~\cite{Coron1999}}
Essa contramedida pode ser aplicada seguindos os passos apresentados a seguir: antes da primeira multiplicação escalar com o escalar $k$ é necessário gerar um ponto aleatório $R$ no subgrupo. Então, pre-compute e armazene $S = [k]R$. No início de cada operação escalar calcule $T \rcv P + R$ e $Q \rcv [k] T$. Atualize $R$ e $S$ da seguinte maneira: $R \rcv (-1)^t 2R$ e $S \rcv (-1)^t 2S$, $t \in \{0,1\}$ escolhido de forma aleatória. Por fim, retorne $W = Q - S$.

O custo encontrado nessa contramedida é de 2 ECADD, 2 ECDBL, e memória SRAM para valores temporários. Provavelmente também é utilizada memória não volátil para armazenar $R$ e $S$. \textit{Point Blinding} protege contra SVA horizontal \cite{MurdicaGuilley2012}, RPA \cite{Goubin2003}, e ZPA \cite{AkishitaTakagi2003}, mas é vulnerável à OTA \cite{BatinaChmielewski2014}

\begin{comment}
\erick[inline]{Lucas: traduzir e escrever em texto corrido.}

Before the first scalar multiplication with scalar $k$:
\begin{itemize}
	\item Generate random point $R$ in the subgroup.
	\item Precompute and store $S = [k]R$.
\end{itemize}

At the beginning of each scalar multiplication:
\begin{itemize}
	\item Compute $T \rcv P + R$.
	\item Compute $Q \rcv [k] T$.
	\item Update $R$ and $S$: $R \rcv (-1)^t 2R$ and $S \rcv (-1)^t 2S$, $t \in \{0,1\}$ is randomly chosen.
	\item Return $W = Q - S$.
\end{itemize}

\noindent  \underline{Cost}: 2 ECADD, 2 ECDBL, SRAM memory for temps. Probably also writable non-volatile memory to store $R$ and $S$.\\
\noindent \underline{Effectiveness}: protects against horizontal SVA~\cite{MurdicaGuilley2012}, RPA~\cite{Goubin2003} and ZPA\cite{AkishitaTakagi2003}. Vulnerable to OTA~\cite{BatinaChmielewski2014}.
\end{comment}


% Definitions ==========================================================
% RPA = Refined Power Analysis. Intermed. point with one of the coords zero.
% ZPA = Zero Value Point Attacks. Itermed. value (e.g., field elem), is zero
% SVA = Same Value Analysis. Generalization of ZPA when the intermediate value has a special value (not necessarily 0), that can be easily distinguished.
%==========================================================
\subsubsection{Scalar Splitting (SS)~\cite{ClavierJoye2001}}

Seja $k$ o escalar original de $n$ bits. Gere um inteiro $k_1 < k$ aleatoriamente e faça $k_2\rcv k - k_1$. Calcule $Q\rcv [k_1]P$, $T\rcv[k_2]P$ e $R\rcv Q + T$. 

\noindent \underline{Custo}: 1 ECSM de $n$ bits e 1 ECADD. Pode ser reduzido se for empregado truque de Shamir para multiplicação escalar dupla (versão regular)~\cite{CietJoye2003}.\\
\noindent \underline{Eficácia}: Resistente à TA, SPA clássico e CPA clássico.\\
\noindent \underline{Variantes}: Euclidean splitting~\cite{CietJoye2003} e multiplicative splitting~\cite{TrichinaBelleza2003}, ambas com a mesma eficácia da versão original, também conhecida como additive splitting.

\begin{comment} % === Ataques SCA contra ECC devem ser do tipo single-trace  ===
5.	Explicar porque no contexto de PKC (RSA e ECC) não fazem sentido ataques que envolvem mais de um trace, como p.ex., DPA.
\end{comment}


\begin{comment}  % === ESTRUTURA ORIGINAL ===
\subsubsection{Ataques baseados em templates}
\subsubsection{Ataques horizontais baseados em cross-correlation}
\subsubsection{Ataques horizontais não-supervisionados baseados em clustering}

\subsubsection{Aplicação de contramedidas em algoritmo esquerda para direita inseguro}
\subsubsection{Implementações de tempo constante}
\subsubsection{Implementações resistentes ao SPA}
\subsubsection{Impacto das contramedidas no desempenho}

\subsection{Eficácia de implementação de tempo constante}
\subsubsection{Outros métodos para inviabilizar ataques por tempo}
\end{comment}

\subsection{Ataque SPA à alg. Montgomery Ladder com SR}

\erick[inline]{Descrever idéia de online template attack (OTA)~\cite{BatinaChmielewski2014}}

\subsection{Ataque SPA à alg. ECSM atômico com SR}
\erick[inline]{Explicar como online template attack (OTA)~\cite{BatinaChmielewski2014} pode ser aplicado neste caso}

\subsection{Ataque template SPA à alg. Montgomery Ladder com SR + CRR}

\erick[inline]{Copy-and-paste de partes do meu paper no SAC 2016.}

\subsection{Ataque HCA à alg. Montgomery Ladder c/ SR + CRR}
\erick[inline]{Copy-and-paste meus textos sobre HCA.}

\subsubsection{Horizontal Attacks (HA)}
% source: RSC-intern-plan

Ataques horizontais (HA) são uma metodologia para ataques por canal lateral cujos alvos são as principais operações criptográficas em protocolos baseados em RSA e ECC, a exponenciação modular e a multiplicação escalar, respectivamente. Em teoria, tais ataques permitem recuperar os bits do expoente/escalar secreto através da análise de traces individuais, isto é, apenas um único trace obtido do alvo é suficiente, portanto são eficazes contra implementações protegidas por contramedidas como SR, CR, PB e SS.

Um requisito básico dos ataques horizontais é o conhecimento do algoritmo de multiplicação escalar.
De posse de tal informação, o atacante pode escolher, dentre outros, os seguintes distinguishers\erick{verificar se distinguishers foi definido previamente ou um termo em portugues foi usado previamente}: SPA, distancia euclidiana, horizontal correlation analysis, horizontal collision-correlation, horizontal cross-correlation ou clustering.

A maioria das formas de ataques horizontais requer pré-processamento avançado dos traces, caracterização e avaliação de vazamento antes da aplicação de distinguishers. Os principais problemas da abordagem horizontal são de que extrair o vazamento a partir de um único trace tipicamente apresenta fortes limitações e desafios, devido ao ruído nos traces e a indisponibilidade de informação rotulada. Em particular, métodos de avaliação de vazamento, como o TVLA~\cite{Goodwill2011}, requerem amostras rotuladas e isso não é possível quando a contramedida SR é aplicada.

Métodos baseados em aprendizado não supervisionada, recentemente aplicados para resolver tais limitações, têm se mostrado capazes de produzir resultados práticos. Heyszl et al~\cite{Heyszl2013} propôem aplicar classificação por clustering à um único trace para possibilitar a identificação de classes específicas de operações; este método funciona bem para medições com baixo ruído e requer uma estação de EM composta de múltiplas sondas. Em Perin et al's~\cite{Perin2014}, os autores consideram uma abordagem heurística baseada na diferença de médias para a seleção de pontos de interesse. Além disso, ambas soluções usam um único trace como entrada para a etapa de avaliação de vazamento, o qual pode ter sido muito afetado por uma grande quantidade de ruído. Perin e Chmielewski~\cite{PerinChmielewski2015} fornecem uma metodologia para ataques horizontais baseados em clustering que foca em corrigir as deficiências dos trabalhos mencionados anteriormente.

\subsection{Ataques template versus Ataques horizontais}
\erick[inline]{Ataques template versus HCA: vantagens e desvantagens de cada um.}

\noindent \textbf{Precondições e limitações dos ataques baseados em template}: Ataques baseados em template são os mais poderosos ataques do tipo SCA, segundo a teoria da informação~\cite{ChariRaoRohatgi2003}. No entanto, ataques baseados em template só podem ser realizados quando a contramedida SR não é aplicada ou quando esta pode ser desabilitada durante a fase de criação de templates (profiling), caso contrário os templates não podem ser criados. Uma outra limitação deste tipo de ataque é de que dispositivos diferentes, mesmo que sejam do mesmo modelo, mesmo lote, etc., têm imperfeições únicas resultantes do processo de fabricação as quais resultam em diferenças no consumo de potência e radiação eletromagnética. Tais diferenças podem ser grandes o suficiente de modo que os templates gerados a partir dos traces provenientes do dispositivo de profiling não sejam bons modelos do vazamento observado no dispositivo alvo do ataque, assim reduzindo a taxa de sucesso do ataque~\cite{ElaabidGuilley2012}.


\noindent \textbf{Aplicabilidade}. Até então estes ataques só foram demonstrados em CPUs embarcadas de 8, 16 e 32 bits, devido ao alto nível de SNR (Signal-to-Noise Ratio) que pode ser obtido na medição no consumo de potência e EM nestes dispositivos. Quando o SNR é baixo, além de haver pouco vazamento de dados (data-leakage) explorável do valor da chave ou valores intermediários derivados deste, o alinhamento dos subtraces torna-se também inviável, devido a inexistência de intervalos próximos da ocorrência da operação alvo em que as amostras tem valores idênticos ou semelhantes em todos os subtraces.

